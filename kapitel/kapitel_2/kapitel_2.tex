% !TEX root = ../main.tex
\newpage
\section{Spaß mit LaTeX}
\subsection{Mathe}
\label{sec:}

Expand $(a+b)^n$:
\newcount\mycntr

  \begin{center}
    \mycntr=0
    \loop\advance\mycntr by 1
    \ifnum\mycntr<20
      $(a\hskip\mycntr pt +\hskip\mycntr pt b)^n$\\
    \repeat
  \end{center}

If~~$\displaystyle\lim_{x\rightarrow8}\frac{1}{x{-}8}=\infty$
~~then~~$\displaystyle\lim_{x\rightarrow5}\frac{1}{x{-}5}=\rotatebox{90}{5}$
\subsection{TikZ}
\label{sec:tikz}
\resizebox{\textwidth}{!}{%
\tikz{
    \draw(-10,-10)rectangle+(20,20);
    \foreach\x/\y in{
        -1/ 1,  0/ 1,  1/ 1,
        -1/ 0,  0/ 0,  1/ 0,
        -1/-1,  0/-1,  1/-1,
          -.5/ .5, .5/ .5,
          -.5/-.5, .5/-.5
    }{
        \begin{scope}
            \tikzset{shift={(\x*6.6,\y*6.6)},xscale=(-1)^(\x+\y)}
            \pgflowlevelsynccm
            \foreach\j in{1,...,15}{
                \draw[line width=6mm,
                    dash pattern={on13.408ptoff13.408pt},
                    dash phase=\j*13.408pt]
                    circle(3);
                \draw[line width=6mm,white,
                    dash pattern={on13.408ptoff13.408pt},
                    dash phase=(\j+1)*13.408pt]
                    circle(3);
                \foreach\i in{1,...,20}{
                    \tikzset{rotate=\i*18+\j*9}
                    \fill[yellow!80!black]
                        (3,0)ellipse[x radius=3mm,y radius=1.5mm];
                    \tikzset{rotate=9}
                    \fill[blue]
                        (3,0)ellipse[x radius=3mm,y radius=1.5mm];
                }
                \tikzset{scale=.81818}
                \pgflowlevelsynccm
            }
        \end{scope}
    }
}
}
