%!TEX root = master.tex
\section*{Einleitung}
Das Thema \glqq Eurokrise\grqq \ ist nun seit bald 10 Jahren ein ständiges Thema, dass die Menschen beschäftigt -- und warum auch nicht, geht es doch um sehr viel, und auch die sonstigen Umstände unserer Welt sind nicht gerade entspannend. Und der Euro alleine ist immerhin nicht eine simple Währung, er wird gerne auch als Kleber (ANDERS; ZITAT/VGL) gehandelt, der den Bestand der europäischen Union maßgeblich unterstützt. Seit 2010 wird häufig über Rettungsmaßnahmen gesprochen, meist wird der
umgangssprachliche Begriff \glqq Euro-Rettungsschirm\grqq\ genutzt, der eigentlich ein Sammelbegriff einiger Maßnahmen ist, die im Zuge der Krise ergriffen wurden, aber eine Zahl ist mir dabei immer im Gedächtnis geblieben: 190 Mrd \euro{}, also 190.000.000.000\euro{}, der Anteil, den Deutschland im \gls{ESM} hat.
Weiterhin ist es jedoch, abseits der üblichen politischen Debatte, meines Erachtens nach Still um das Thema geworden, vielleicht habe ich aber auch irgendwann eben wegen der politischen Debatte abgeschaltet.

Als ich mich also für das Thema entschied, wollte ich bewusst ein für mich eher in \glqq Vergessenheit\grqq \ geratenes Thema auf den aktuellen Stand auffrischen.
Zum Zeitpunkt der Entstehung dieser kurzen Einleitung, die als persönlicher Rahmen zu der Arbeit dienen soll, hatte ich mich voller Elan telefonisch mit der Bundesbankfiliale in Düsseldorf in Verbindung gesetzt und gefragt, ob man mich von dort aus eventuell mit etwas Material versorgen kann um entsprechend damit arbeiten zu können. In meiner leichtsinnigen Vorstellung war das einfach und logisch um an Daten zu kommen dieser Art, betrifft es doch jeden Bürger der EU, für mich zuständig ist aber
in erster Instanz nun mal eine deutsche Institution. Mittlerweile weiß ich, dass es gar nicht so einfach ist, Aussagen über \glqq Performance \grqq \ zu erhalten, und ich möchte mich dem Feld also vorsichtig nähern und objektiv Anhand von Fakten, die wir vergleichen und analysieren können, eine Annahme darüber wagen, ob sich die Stabilität des \gls{Euro} in den letzten Jahren verbessert hat, vor allem im Bezug auf die Länder, die am schwersten betroffen waren.

Zum Schluss also die Antwort als kleines Zitat:
\begin{quote}
  \enquote{\textit{Dafür werden Sie von der Bundesbank in der Form niemals einen Kommentar erhalten.}}
\end{quote}
\hfill Anonymer Mitarbeiter der Bundesbank

Ich hatte mir das einfacher vorgestellt\dots
