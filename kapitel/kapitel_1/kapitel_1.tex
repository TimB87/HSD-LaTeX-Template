\newpage
\section{Begriffsabgrenzung}
Im folgenden möchte ich definieren, welche Maßnahmen auf welches Aufgabe der \gls{EU} es zu beurteilen gibt.

\subsection{Eurokrise}
Unter dem Begriff Eurokrise wird eine Reihe verschiedener Krisen zusammengefasst, die alle in der \gls{EU} schwelten und etwa zeitgleich das kritische Maß erreichten. Die Ursachen waren dabei jeweils unterschiedlich: in Griechenland handelte es sich vorwiegend um ein betrügerisches Staatsversagen (ZITAT), in Spanien und Irland ist eine Immobilienblase geplatzt (ZITAT), was in Irland dazu führte, das ein sowieso viel zu großes Bankwesen (ZITATE SUCHEN) kollabierte, letztes wurde auch dem Land Zypern zum Unheil, welches überwiegend durch seine volkswirtschaftliche Nähe zu Griechenland stark in Mitleidenschaft gezogen wurde (Zitat).

Im Falle Griechenlands war der EU seit 2004 bekannt, dass das Euroland hoch verschuldet war. Ein Bericht der Europäischen Kommission im Jahr 2010 kommt sogar zu dem Schluss, dass die statistischen Behörden in Griechenland die Höhe des Defizites in den Jahren 1997 bis 2003 unterschlagen hätten, vorher hatte Ende 2009 der griechische Ministerpräsident Giorgos Papandreou ein Staatsdefizit in Höhe von 12,5\% eingeräumt, nach vorherig 3,7\%
gemeldeten\footcite[vgl.][]{europaische_kommision_bericht_2010}
%Zitat http://ec.europa.eu/eurostat/documents/4187653/6406122/COM_2010_bericht_Griechenland.pdf/ Seite 30 \glqq … daraus ergab sich, dass die griechischen Statistikstellen zwischen 1997 und 2003 die Defizit- und Schuldenstandszahlen nicht korrekt gemeldet hatten. \grqq oder ich mach das als vergleich

In Spanien und Irland hingegen ist eine Immobilienblase geplatzt. Im Zuge jahrelangen starken Wachstums haben sich die Privathaushalte, häufig mit Geld aus dem Ausland, überschuldet und sind dann kollabiert. Das hat in Irland zusätzlich eine Bankenkrise hervorgerufen, da dort bis zu 80\% der neu vergebenen Kredite für Immobilien aufgenommen wurden.
Zusätzlich haben in Portugal und Italien ebenfalls marode politische Praktiken dazu geführt, dass sich eine Staatsverschuldung verschleppte. Italiens Staatsverschuldung lag 2018 noch bei 129,75\% in Relation zum Bruttoinlandsprodukts, zu Beginn der Krise 2009 immerhin schon bei 112,55\% (IWF, STATISTA).
Zypern war durch seine enge volkswirtschaftliche Verflechtung mit Griechenland in Mitleidenschaft gezogen wurden, das Rating des Landes wurde herabgestuft und ein Unglück auf einem Militärstützpunkt stürzten das Land in ein politisches Chaos (quelle?).

Im Zuge dessen haben die Märkte verstimmt reagiert und angefangen, Ländern zu misstrauen, so dass der Preis für Kredite stieg. Griechenland war somit nur das erste Land, dass um Hilfe fragen musste, um weiterhin Zugang zum Markt zu haben\footcite[vgl.][]{esm_history_nodate}% Quelle (https://www.esm.europa.eu/about-us/history#context)

\subsection{Europäischer Finanzstabilisierungsmechanismus}
Bei dem \gls{EFSM} handelt es sich um einen Vertrag, der auf Basis des Art. 122 AEU-Vertrag die \gls{EU} Kommision direkt ermächtigt, Kredite an Mitgliedsstaaten auszuzahlen, die in Schieflage geraten sind.
Die Gelder stammen dabei aus dem allgemeinen Haushaltsmitteln der Union und durchbricht somit die Regelung, dass sich die \gls{EU} nicht selbst verschulden darf. Das Risiko des Zahlungsausfalls liegt hier also bei allen Mitgliedsstaaten, also nicht nur jenen, die den \gls{Euro} als Zahlungsmittel haben.

Mit dem \gls{EFSM} stand also in erster Linie ein Instrument zur Verfügung, dass Liquidität in das Krisenland bringen konnte. Das Risiko trugen dabei alle Staaten solidarisch, auch jene, die nicht der Währungsunion beigetreten sind, da der gesamte Haushalt der Union als Sicherheitspfand eingetragen wurde. (QUELLE)

\subsection{Europäische Finanzstabilisierungsfaszilität}
Bei dem \gls{EFSF} handelt es sich, anders als beim \gls{EFSM}, um einen privatrechtlichen Vertrag zwischen den Mitgliedstaaten der Union, die der Währungsunion beigetreten sind. Diese Aktiengesellschaft wurde am 27.8.2010 nach luxemburgischem Recht, im Zuge einer Sondersitzung zur \glqq{} Eurokrise \grqq, gegründet und hatte die Aufgabe, zügig Kredite in die Krisenländer, damals Griechenland, Portugal und Irland zu bringen. Es handelte sich um eine vorübergehende Maßnahme, die nun so ziemlich
durch den \gls{ESM} abgelöst wurde. \footcite[vgl.][]{} %(http://www.efsf.europa.eu/about/operations/index.htm)
Der \gls{EFSF} hat insgesamt 174,6 Mrd. \euro{} an die drei genannten Länder geliefert und ist planmäßig nun weiterhin nur verwaltend tätig, kann aber keine neuen Kredite vergeben.

\subsection{Europäischer Stabilitätsmechanismus}
Der \gls{ESM} ist die dauerhafte Implemention der Mechanismen des \gls{EFSM} und \gls{EFSF}, wurde ähnlich wie der \gls{EFSF} als völkerrechtlicher Vertrag geschlossen und löste die vorherigen Maßnahmen im Zuge immer weiter ab.
Nur Griechenland erhielt noch einmal durch den \gls{EFSF} im Jahr 2014 ein zusätzliches Hilfspaket, durchbrach also oben genannte Zahlungsunfähigkeit, die Verwaltung des Pakets wurde jedoch durch dem \gls{ESM} übernommen.

Für den \gls{ESM} wurde vorher in der \gls{EU} die rechtliche Grundlage geschaffen, und zwar mit einem Zusatz in Art. 136 AEU-Vertrag und wurde dann in Deutschland im September 2012 rechtskräftig.

Der \gls{ESM} verfügt über ein gezeichnetes Kapital in Höhe von 704,80 Mrd. \euro{}, das eingezahlte Kapital beträgt 80,55 Mrd \euro{}. Die Finanzierung passiert über den offenen Finanzmarkt.
\begin{table}[H]
\begin{tabular}{|l|l|l|l|l|}
\hline
\rowcolor[HTML]{C0C0C0} 
{\color[HTML]{000000} Datum} & {\color[HTML]{000000} \begin{tabular}[c]{@{}l@{}}EFSF\\ in Mrd.\end{tabular}} & {\color[HTML]{000000} Fälligkeit} & {\color[HTML]{000000} \begin{tabular}[c]{@{}l@{}}EFSM\\ in Mrd.\end{tabular}} & {\color[HTML]{000000} Laufzeit (Jahre)} \\ \hline
12.01.2011 & & & 2,0 & 13 \\ \hline
12.01.2011 & & & 1,0 & 19 \\ \hline
12.01.2011 & & & 2,0 & 25 \\ \hline
01.02.2011 & 1,9 & 01.08.2032 & &  \\ \hline
01.02.2011 & 1,7 & 01.02.2033 & & \\ \hline
24.03.2011 & & & 2,4 & 14 \\ \hline
24.03.2011 & & & 1,0 & 22 \\ \hline
31.05.2011 & & & 3,0 & 10 \\ \hline
29.09.2011 & & & 2,0 & 15 \\ \hline
06.10.2011 & & & 0,5 & 22 \\ \hline
10.11.2011 & 0,9 & 01.08.2030 & & \\ \hline
10.11.2011 & 2,1 & 25.07.2031 & & \\ \hline
15.12.2011 & 1,0 & 01.08.2029 & & \\ \hline
12.01.2012 & 1,3 & 01.08.2029 & & \\ \hline
16.01.2012 & & & 1,5 & 30 \\ \hline
19.01.2012 & 0,5 & 01.07.2034 & & \\ \hline
05.03.2012 & & & 3,0 & 20 \\ \hline
03.04.2012 & 2,8 & 01.08.2031 & & \\ \hline
03.07.2012 & & & 2,3 & 15 \\ \hline
30.10.2012 & & & 1,0 & 15 \\ \hline
02.05.2013 & 0,8 & 01.08.2029 & & \\ \hline
18.06.2013  & 1,6 & 15.11.2042 & & \\ \hline
27.09.2013 & 1,0 & 27.09.2034 & & \\ \hline
04.12.2013 & 2,3 & 04.12.2033 & & \\ \hline
25.03.2014 & & & 0,8 & 10 \\ \hline
\rowcolor[HTML]{C0C0C0} 
Summe & 17,7 & & 22,5 & \\ \hline
\end{tabular}%
\caption{Die Auszahlungen des EFSF und EFSM im Überblick}
\label{Tabelle 1}
\end{table}

\begin{table}[H]
\begin{tabular}{|l|l|l|l|l|}
\hline
\rowcolor[HTML]{C0C0C0} 
\begin{tabular}[c]{@{}l@{}}Zugesagte \\ Programmvolumina (bis zu… Mrd. \euro{})\end{tabular} & \begin{tabular}[c]{@{}l@{}}Auszahlung\\ in Mrd. \euro{}\end{tabular} &       & \multicolumn{2}{l|}{\cellcolor[HTML]{C0C0C0}\begin{tabular}[c]{@{}l@{}}Rückzahlungen\\ in Mrd. \euro{}\end{tabular}} \\ \hline
\rowcolor[HTML]{EFEFEF} 
\multicolumn{3}{|l|}{\cellcolor[HTML]{EFEFEF}} & \multicolumn{2}{l|}{\cellcolor[HTML]{EFEFEF}500,0} \\ \hline
Spanien & 41,3 & -17,6 & \multicolumn{2}{l|}{23,7} \\ \hline
Zypern & 6,3 & -     & \multicolumn{2}{l|}{6,3} \\ \hline
Griechenland & 61,9 & -2,0  & \multicolumn{2}{l|}{59,9} \\ \hline
Summe & 195 & -19,6 & \multicolumn{2}{l|}{89,9} \\ \hline
\rowcolor[HTML]{C0C0C0} 
\multicolumn{3}{|l|}{\cellcolor[HTML]{C0C0C0}Verbleibendes ESM-Ausleihvolumen}                                                                                              & \multicolumn{2}{l|}{\cellcolor[HTML]{C0C0C0}410,1}                                                                   \\ \hline
\end{tabular}
\caption{Die Leistungen des ESM}
\label{Tabelle 2}
\end{table}
Insgesamt stehen dem \gls{ESM} mittlerweile folgende Werkzeuge zur Verfügung:\begin{itemize}
  \item Darlehen im Rahmen eines makroökonomischen Anpassungsprogramms\begin{itemize}
    \item In Anwendung in Irland, Portugal, Griechenland und Zypern
  \item Die Voraussetzung ist die Akzeptanz für die Umsetzung eines durch die \gls{EU}, \gls{EZB} und wo erforderlich der \gls{IWF} erstellten, mikroökonomischen Reformpaketes
  \item Ist zur Unterstützungen bei Zahlungsunfähigkeit und Marktzugangsverlust gedacht\end{itemize}
  \item Direktmarktkäufe \begin{itemize}
    \item bisher ungenutzt
    \item keine weiteren Voraussetzungen
    \item \end{itemize}
  \item Sekundärmarktkäufe\begin{itemize}
    \item bisher ungenutzt
    \item Für Mitglieder, die bereits eine andere Maßnahme des \gls{ESM} erhalten werden individuelle Konditionen erstellt\end{itemize}
  \item Vorsorgekredite\begin{itemize}
    \item bisher ungenutzt
    \item \end{itemize}
  \item Darlehen für die indirekte Rekapitalisierung von Banken\begin{itemize}
    \item In Anwendung in Spanien\end{itemize}
  \item Direkte Rekapitalisierung von Institutionen\begin{itemize}
    \item bisher ungenutzt\end{itemize}
\end{itemize}\footcite{staab_european_2013}
https://www.esm.europa.eu/assistance/lending-toolkit

\subsection{Die Hilfspakete in Zahlen}
\subsubsection{Leistungen des EFSF und EFSM}
%https://www.bundesfinanzministerium.de/Content/DE/Standardartikel/Themen/Europa/Stabilisierung_des_Euro/europaeische-finanzhilfen-efsf-efsm.html
