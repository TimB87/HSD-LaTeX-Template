% !TEX root = ../main.tex
%%% general page related settings
\documentclass[12pt,%
  %draft,%
  overfullrule=true,%
  oneside,%
  titlepage,%
  listof=totoc,%
  twoside=semi,%
  bibliography=totoc]{scrartcl} % https://www.ctan.org/pkg/scrartcl

\usepackage[automark,headsepline]{scrlayer-scrpage} % https://www.ctan.org/pkg/scrlayer-scrpage

\usepackage[a4paper,%
  left=3cm,%
  right=2.5cm,%
  top=2cm,%
  bottom=2cm]{geometry}

\usepackage{setspace} % 1,5 cm Zeilenabstand
\onehalfspacing{}

%\setlength{\parindent}{0mm} % https://komascript.de/faq_parindent
%\setlength{\parskip}{0.8em plus 0.5em minus 0.3em}
%\sloppy %Abstände variieren

%%% language settings
\usepackage{hyphsubst} % https://www.ctan.org/pkg/hyphsubst
\HyphSubstIfExists{ngerman-x-latest}{%
\HyphSubstLet{ngerman}{ngerman-x-latest}}{}
\usepackage[ngerman]{babel} % https://www.ctan.org/pkg/babel
\usepackage[babel,german=guillemets]{csquotes} % https://www.ctan.org/pkg/csquotes

%%% font settings
%\usepackage[utf8]{inputenc} % not needed with luatex
\usepackage{fontspec} % luatex needs this one
\setmainfont{Arial} % we even get to use Arial, woohoo..
\renewcommand\familydefault{\sfdefault} %defaults to helvetica
%%% if you don't or can't use luatex, use this
%% helvet is alternative to Arial for tex, or it's actually the original font
%% with minimal differences
%\usepackage[scaled=0.9]{helvet} % use together with courier package to avoid pixelated fonts
%\usepackage{courier}

% math related settings
\usepackage{amsmath} % https://www.ctan.org/pkg/amsmath
\usepackage{unicode-math} % https://www.ctan.org/pkg/unicode-math
\setmathfont{texgyrepagella-math.otf}
%\usepackage{mathptmx} % https://www.ctan.org/pkg/mathptmx -- obsolet
%\usepackage{nicefrac} % The pack­age type­sets fractions “nicely”
\usepackage{marvosym} % https://www.ctan.org/pkg/marvosym
\usepackage[hang, multiple]{footmisc} % must be placed above hyperref to avoid warnings
\setlength{\footnotemargin}{1em}
\usepackage[colorlinks=true,%
  breaklinks=true,%
  linkcolor=black,%
  citecolor=blue,%
  menucolor=black,%
  urlcolor=black]{hyperref} % adds new commands to include hyperlinks

%%% additional packages
\usepackage{ragged2e} % https://www.ctan.org/pkg/ragged2e
%\usepackage{footnote} % https://www.ctan.org/pkg/footnote
%\usepackage{fancybox} % https://www.ctan.org/pkg/fancybox

%%% table packages
\usepackage[table,xcdraw]{xcolor} % https://www.ctan.org/pkg/xcolor
\usepackage{multirow} % https://www.ctan.org/pkg/multirow
%\usepackage{array} % https://www.ctan.org/pkg/array
%\usepackage{colortbl} % https://www.ctan.org/pkg/colortbl
\usepackage{microtype} % https://www.ctan.org/pkg/microtype

%%% floats
\usepackage{float} % https://www.ctan.org/pkg/float
\usepackage{caption} % https://www.ctan.org/pkg/caption
\usepackage{graphicx} % https://www.ctan.org/pkg/graphicx

%%% list related
\usepackage{enumitem} % https://ctan.org/pkg/enumitem
%\usepackage{mdwlist} % https://www.ctan.org/pkg/mdwlist

%%% Glossary
\usepackage[%
  %sort=word,%
  xindy%
  ]{glossaries-extra} % https://www.ctan.org/pkg/glossaries
\setabbreviationstyle{long-short}
\loadglsentries{meta/gls.tex} % definitions go there
% index w/ xindy
\usepackage{makeidx} % https://www.ctan.org/pkg/makeidx
\usepackage{idxlayout} % https://www.ctan.org/pkg/idxlayout
\makeindex
%%% List of Abbreviations
\usepackage[refpage,intoc]{nomencl}
\makenomenclature
\makeatletter
\def\thenomenclature{%
  \@ifundefined{chapter}%
  {
    \section*{\nomname} % do not give it a section number
    \if@intoc\addcontentsline{toc}{section}{\nomname}\fi%
  }%
  {
    \chapter*{\nomname} % do not give it a chapter number
    \if@intoc\addcontentsline{toc}{chapter}{\nomname}\fi%
  }%

  \nompreamble
  \list{}{%
    \labelwidth\nom@tempdim
    \leftmargin\labelwidth
    \advance\leftmargin\labelsep
    \itemsep\nomitemsep
    \let\makelabel\nomlabel}}
\makeatother

%%% Biblatex
\usepackage[%
  backend=biber,%
  style=numeric,%
  citestyle=authoryear,%
  %sortcase=true,%
  date=iso,%
  seconds=true,%
  urldate=iso,%
  %dashed=false,%
  %doi=false,%
  isbn=true,%
  %mergedate=false,%
  autocite=footnote,%
  sortlocale=de]{biblatex}
\input{skripte/modsBiblatex}

% don't use fancyhdr to change headers/footers

% define headers and footers
%\usepackage{fancyhdr}
%\pagestyle{fancy}
%\fancyhf{}
%\fancyhead[C]{-\ \thepage\ -} % centered page numbering
%\fancyhead[L]{\leftmark}      % no footer
%\renewcommand{\headrulewidth}{0.4pt}
% footnote style
%\let\footnoteOld\footnote{}
%\renewcommand{\footnote}[1]{%
%  \linespread{1.0}%
%  \footnoteOld{#1}%
%  \linespread{1.2}%
%}
%Fußnoten auf Schriftgröße 10

% use "scrheadings"
\clearpairofpagestyles
\cfoot[\pagemark]{\pagemark}
\lehead{\headmark}
\rohead{\headmark}
\pagestyle{scrheadings}

\setkomafont{footnote}{\footnotesize{10}}

% Abkürzungsverzeichnis mit glossaries
%\makenoidxglossaries{} % w/o xindy
\makeglossaries % w/ xindy

%%% proper source code display
% disable those lines if you don't display any code
\usepackage{listings} % https://www.ctan.org/pkg/listings
% Colors for listings
\definecolor{mauve}{rgb}{0.58,0,0.82}
\definecolor{dkgreen}{rgb}{0,0.6,0}
\lstset{% general code settings
  numbers=left,%
  numberstyle=\tiny,%
  numbersep=5pt,%
  breaklines=true,%
  showstringspaces=false,%
  frame=l ,%
  xleftmargin=5pt,%
  xrightmargin=5pt,%
  basicstyle=\ttfamily\scriptsize,%
  stepnumber=1,%
  keywordstyle=\color{blue},% keyword style
  commentstyle=\color{dkgreen},% comment style
  stringstyle=\color{mauve}% string literal style
}
