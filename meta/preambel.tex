% !TEX root = ../../main.tex
%%% general page related settings
%#{{{
\documentclass[12pt,%
  %draft,%
  %overfullrule=true,% helps debugging overfull hbox warnings
  oneside,%
  titlepage,%
  listof=totoc,%
  bibliography=totoc]{scrartcl} % https://www.ctan.org/pkg/scrartcl

\usepackage[automark,headsepline]{scrlayer-scrpage} % https://www.ctan.org/pkg/scrlayer-scrpage

\usepackage[a4paper,%
  left=3cm,%
  right=2.5cm,%
  top=2cm,%
  bottom=2cm]{geometry}

%%% don't use fancyhdr to change headers/footers
%%% this is left as a warning
% define headers and footers
%\usepackage{fancyhdr}
%\pagestyle{fancy}
%\fancyhf{}
%\fancyhead[C]{-\ \thepage\ -} % centered page numbering
%\fancyhead[L]{\leftmark}      % no footer
%\renewcommand{\headrulewidth}{0.4pt}
% footnote style
%\let\footnoteOld\footnote{}
%\renewcommand{\footnote}[1]{%
%  \linespread{1.0}%
%  \footnoteOld{#1}%
%  \linespread{1.2}%
%}

%%% use "scrheadings" instead with a KOMA class
\clearpairofpagestyles
\rehead{\headmark} % always print header on the right corner of the header
\rohead{\headmark} % --------------------------"--------------------------
\pagestyle{scrheadings}
\cfoot[\pagemark]{\pagemark} % always print the page number centered in the footer
\KOMAoptions{headsepline=.5pt}

\usepackage{setspace} % 1,5 cm Zeilenabstand
\onehalfspacing{}

%%% don't modify parindent! see attached linked for an explanation why
%\setlength{\parindent}{0mm} % https://komascript.de/faq_parindent
%\setlength{\parskip}{0.8em plus 0.5em minus 0.3em}
%\sloppy

%%%end general page related settings
%#}}}

%%% language settings
%{{{
\usepackage{hyphsubst} % https://www.ctan.org/pkg/hyphsubst
\HyphSubstIfExists{ngerman-x-latest}{%
\HyphSubstLet{ngerman}{ngerman-x-latest}}{}
\usepackage[ngerman]{babel} % https://www.ctan.org/pkg/babel
\usepackage[babel,german=guillemets]{csquotes} % https://www.ctan.org/pkg/csquotes
%}}}

%%% font settings
%{{{
%\usepackage[utf8]{inputenc} % not needed with luatex
\usepackage{fontspec} % luatex needs this one
\setmainfont{Arial} % we even get to use Arial, woohoo..
\setkomafont{footnote}{\footnotesize{10}} % footnote font size is 10pt
%\renewcommand\familydefault{\sfdefault} %defaults to helvetica
%%% if you don't or can't use luatex, use this
%% helvet is an available alternative to Arial, it's actually the original font
%% with minimal differences you probably don't need to worry about
%\usepackage[scaled=0.9]{helvet} % use together with courier package to avoid pixelated fonts
%\usepackage{courier}
\usepackage{microtype} % https://www.ctan.org/pkg/microtype
%}}}

% math related settings
% {{{
\usepackage{amsmath} % https://www.ctan.org/pkg/amsmath
\usepackage{unicode-math} % https://www.ctan.org/pkg/unicode-math
\setmathfont{texgyrepagella-math.otf}
%\usepackage{mathptmx} % https://www.ctan.org/pkg/mathptmx -- obsolet
%\usepackage{nicefrac} % The pack­age type­sets fractions “nicely”
\usepackage{marvosym} % https://www.ctan.org/pkg/marvosym
%}}}

%%% references
% {{{
% footmisc must be placed above hyperref to avoid warnings
\usepackage[hang, multiple]{footmisc} % https://www.ctan.org/pkg/footmisc
\setlength{\footnotemargin}{1em}
%\usepackage{varioref} % https://www.ctan.org/pkg/varioref
\usepackage[unicode,%
  colorlinks=true,%
  breaklinks=true,%
  linkcolor=black,%
  citecolor=blue,%
  menucolor=black,%
  urlcolor=black]{hyperref} % adds new commands to include hyperlinks
%\usepackage{cleveref} % https://www.ctan.org/pkg/cleveref
%}}}

%%% additional packages
%{{{
%\usepackage{ragged2e} % https://www.ctan.org/pkg/ragged2e
%\usepackage{fancybox} % https://www.ctan.org/pkg/fancybox
\usepackage{xpatch} % https://www.ctan.org/pkg/xpatch
\usepackage{hologo} % https://www.ctan.org/pkg/hologo
%}}}

%%% table packages
%{{{
\usepackage[table,xcdraw]{xcolor} % https://www.ctan.org/pkg/xcolor
\usepackage{multirow} % https://www.ctan.org/pkg/multirow
%\usepackage{array} % https://www.ctan.org/pkg/array
%\usepackage{colortbl} % https://www.ctan.org/pkg/colortbl
%}}}

%%% floats
%{{{
\usepackage{float} % https://www.ctan.org/pkg/float
\usepackage{caption} % https://www.ctan.org/pkg/caption
% hypcap is true by default so [hypcap=true] is optional in \usepackage[hypcap=true]{caption}
%}}}

%%% graphics
%{{{
\usepackage{graphicx} % https://www.ctan.org/pkg/graphicx
\usepackage{tikz} % https://www.ctan.org/pkg/pgf
\usetikzlibrary{trees}
%}}}

%%% list related
%{{{
\usepackage{enumitem} % https://ctan.org/pkg/enumitem
%\usepackage{mdwlist} % https://www.ctan.org/pkg/mdwlist
%}}}

%%% Biblatex
%{{{
\usepackage[%
  backend=biber,%
  style=apa,%
  date=iso,% access date in iso format
  seconds=true,% required by date=iso
  url=true,% prints url, if available
  urldate=iso,% access date in iso format
  dateera=astronomical,% 'date=iso' requires 'dateera=astronomical'
  isbn=true,% print isbn
  doi=false,% omit doi
  autocite=footnote,% automatically use \footcite
  maxcitenames=2, % shorten authors if more than 2
  maxbibnames=999,%
  giveninits=false,% prints a full name if set to false
  %eprint=true,%
  backref=false,% prints backrefs in \printbibliography
  bibencoding=utf8,% tries to encode with UTF-8
  bibwarn=true,% can be disabled, else shows warnings via buildlog
  sortlocale=de% can be deleted if you don't write in german
  ]{biblatex} % https://www.ctan.org/pkg/biblatex
%}}}

%%% glossary/xindy/acronym
%{{{
\usepackage[%
  xindy,% index w/ xindy
  acronym,% acronyms w/ xindy
  toc,%
  %style=tree%
  ]{glossaries-extra} % https://www.ctan.org/pkg/glossaries
%%% abbreviations
\loadglsentries{meta/acro.tex} % acronym definitions go there
\setabbreviationstyle{long-short}

%%% glossary settings
\GlsSetXdyLanguage{german}
\GlsSetXdyCodePage{duden-utf8}
\loadglsentries{meta/gls.tex} % glossary definitions go there
\makeglossaries % w/ xindy
%\makenoidxglossaries{} % w/o xindy

%%% index settings
\usepackage{makeidx} % https://www.ctan.org/pkg/makeidx
\usepackage{idxlayout} % https://www.ctan.org/pkg/idxlayout
\makeindex
%}}}

%%% proper source code display
%{{{
% disable those lines if you don't display any code
\usepackage{listings} % https://www.ctan.org/pkg/listings
% Colors for listings
\definecolor{mauve}{rgb}{0.58,0,0.82}
\definecolor{green}{rgb}{0,0.6,0}
\lstset{% general code settings
  language=TeX,%
  basicstyle=\footnotesize,%
  numbers=left,%
  numberstyle=\tiny,%
  numbersep=5pt,%
  frame=single,%
  breaklines=true,%
  title=\lstname,%
  xleftmargin=5pt,%
  xrightmargin=5pt,%
  basicstyle=\ttfamily\scriptsize,%
  stepnumber=1,%
  captionpos=b,%
  keywordstyle=\color{blue},% keyword style
  commentstyle=\color{green},% comment style
  stringstyle=\color{mauve}% string literal style
}
\usepackage{scrhack} % needed for listings and komascript
%}}}
