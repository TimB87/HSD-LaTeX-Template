% !TEX root = ../main.tex
%%% here we can define our glossary entries
%%% \newglossaryentry{what_we_call_it}{name=name,
%%%  description={description}}

%{{{ kompilieren
\newglossaryentry{kompilieren}{name=kompilieren,
description={[1] Informationen oder Werke zusammenfassen, [2] Softwareentwicklung: ein Programm mit Hilfe eines Compilers in Maschinensprache umwandeln},plural=kompiliert}% not nice, nobody saw that
%}}}
%{{{utf8
\newglossaryentry{utf8}{name=UTF-8,
  description={a variable width character encoding capable of encoding all 1,112,064 valid code points in Unicode using one to four 8-bit bytes}}
%}}}
%{{{pull
\newglossaryentry{pull}{name=pull request,
  description={Pull requests let you tell others about changes you've pushed to a branch in a repository on GitHub. Once a pull request is opened, you can discuss and review the potential changes with collaborators and add follow-up commits before your changes are merged into the base branch}}
%}}}
%{{{dante
\newglossaryentry{dante}{name=Die Deutschsprachige Anwendervereinigung TeX e.V.,
description={Der Zweck des gemeinnützigen Vereins ist die Betreuung von \TeX{}-Nutzerinnen und Nutzern im gesamten deutschsprachigen Raum. Außerdem fördert DANTE e.V. Entwicklungen im Bereich von \TeX{} , \LaTeX{}, \hologo{ConTeXt}, \hologo{LuaTeX}, \hologo{METAFONT}, \hologo{BibTeX}, Schriften personell wie finanziell auf nationaler und internationaler Ebene.},plural=Der Deutschsprachige Anwendervereinigung TeX e.V.}%%}}}
\newglossaryentry{compiler}{name=Compiler,%{{{
  description={Ein Compiler (auch Kompiler; von englisch für zusammentragen bzw. lateinisch compilare ‚aufhäufen‘) ist ein Computerprogramm, das Quellcodes einer bestimmten Programmiersprache in eine Form übersetzt, die von einem Computer (direkter) ausgeführt werden kann.}}%}}}
