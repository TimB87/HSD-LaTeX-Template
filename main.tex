% file: main.tex
\documentclass[12pt,oneside,titlepage,listof=totoc,bibliography=totoc]{scrartcl}
\usepackage[utf8]{inputenc}
\usepackage[ngerman]{babel}
\usepackage[babel,german=quotes]{csquotes}
\usepackage[T1]{fontenc}
\usepackage{fancyhdr}
\usepackage{fancybox}
\usepackage[a4paper, left=3cm, right=2.5cm, top=2cm, bottom=2cm]{geometry}
\usepackage{setspace} % 1,5 cm Zeilenabstand
\onehalfspacing{}
\usepackage{graphicx}
%\usepackage{colortbl}
%\usepackage{array}
\usepackage{float}
\usepackage{footnote}
\usepackage{caption}
\usepackage{enumitem}
\usepackage{amssymb}
\usepackage{mathptmx}
\usepackage[scaled=0.9]{helvet} % Behebt, zusammen mit Package courier, pixelige Überschriften. Ist, zusammen mit mathptx, dem times-Package vorzuziehen. Details: https://latex-kurs.de/fragen/schriftarten/Times_New_Roman.html
\renewcommand\familydefault{\sfdefault}
\usepackage{courier}
\usepackage{amsmath}
\usepackage[table,xcdraw]{xcolor}
\usepackage{marvosym} % Verwendung von Symbolen, z.B. perfektes Eurozeichen
\usepackage[colorlinks=true,linkcolor=black]{hyperref}
\definecolor{darkblack}{rgb}{0,0,0}
\hypersetup{colorlinks=true, breaklinks=true, linkcolor=darkblack, menucolor=darkblack, urlcolor=darkblack}
\fontfamily{ptm}\selectfont
\usepackage{ragged2e}
\usepackage{glossaries-extra}
%\usepackage{glossaries-de}
\setabbreviationstyle{long-short}
\usepackage[official]{eurosym}

% Mehrere Fussnoten nacheinander mit Komma separiert
\usepackage[hang, multiple]{footmisc}
\setlength{\footnotemargin}{1em}

% Pakete für Tabellen
\usepackage{epstopdf}
\usepackage{nicefrac} % Brüche
\usepackage{multirow}
%\usepackage{mdwlist}

\definecolor{dunkelgrau}{rgb}{0.8,0.8,0.8}
\definecolor{hellgrau}{rgb}{0.0,0.7,0.99}
% Colors for listings
\definecolor{mauve}{rgb}{0.58,0,0.82}
\definecolor{dkgreen}{rgb}{0,0.6,0}

% sauber formatierter Quelltext

% Biblatex
\usepackage[
backend=biber,
style=numeric,
citestyle=authoryear,
url=true,
isbn=false,
notetype=footonly,
hyperref=true,
sortlocale=de]{biblatex}

%\usepackage[backend=biber]{biblatex}

%weitere Anpassungen für BibLaTex
\input{skripte/modsBiblatex}


%Bib-Datei einbinden
\addbibresource{literatur/literatur.bib}

% Pfad fuer Abbildungen
\graphicspath{{./}{./abbildungen/}}

%-----------------------------------
% Weitere Ebene einfügen
%\input{skripte/weitereEbene}

%-----------------------------------
% Zeilenabstand 1,5-zeilig
%-----------------------------------

%-----------------------------------
% Absätze durch eine neue Zeile
%-----------------------------------
\setlength{\parindent}{0mm}
\setlength{\parskip}{0.8em plus 0.5em minus 0.3em}

\sloppy %Abstände variieren
\pagestyle{headings}

% Meta informationen
%-----------------------------------
%-----------------------------------
% Meta Informationen zur Arbeit
%-----------------------------------

% Autor
\newcommand{\myAutor}{Tim Biermann}

% Adresse
\newcommand{\myAdresse}{Adersstra\ss{}e 26 \\ \> \> 40215 Düsseldorf}

% Titel der Arbeit
\newcommand{\myTitel}{War der ESM erfolgreich?}

% Betreuer
\newcommand{\myBetreuer}{Prof.\ Dr.\ Mouna Thiele}

% Lehrveranstaltung
\newcommand{\myLehrveranstaltung}{Projektseminar Thesis}

% Matrikelnummer
\newcommand{\myMatrikelNr}{742850}

% Ort
\newcommand{\myOrt}{Düsseldorf}

% Datum der Abgabe
\newcommand{\myAbgabeDatum}{21.03.2019}

% Semesterzahl
\newcommand{\mySemesterZahl}{7}

% Name der Hochschule
\newcommand{\myHochschulName}{Hochschule Düsseldorf}

% Standort der Hochschule
\newcommand{\myHochschulStandort}{Düsseldorf Derendorf}

% Studiengang
\newcommand{\myStudiengang}{Bachelor Business Administration}

% Art der Arbeit
\newcommand{\myThesisArt}{Hausarbeit}

% Zu erlangender akademische Grad
%\newcommand{\myAkademischerGrad}{}

% Firma
%\newcommand{\myFirma}{Mustermann GmbH}



% PDF Meta Daten setzen

\hypersetup{%
    pdfinfo={%
        Title={\myTitel},
        Subject={\myStudiengang},
        Author={\myAutor},
        Build=1.1
    }
}


% Kopfbereich / Header definieren

\pagestyle{fancy}
\fancyhf{}
\fancyhead[C]{-\ \thepage\ -}           % Seitenzahl oben, mittg
%\fancyhead[L]{\leftmark}             % kein Footer vorhanden
\renewcommand{\headrulewidth}{0.4pt}

% Abkürzungsverzeichnis mit glossaries
\makenoidxglossaries{}
%\newabbreviation{}{}{}
\newabbreviation{EU}{EU}{Europäischen Union}
\newabbreviation{EFSF}{EFSF}{Europäische Finanzstabilisierungsfazilität}
\newabbreviation{EFSM}{EFSM}{Europäischer Finanzstabiliserungsmechanismus}
\newabbreviation{EZB}{EZB}{Europäische Zentralbank}
\newabbreviation{EFSI}{EFSI}{Europäische Fonds für strategische Investionen}
\newabbreviation{SWP}{SWP}{Stabilitäts- und Wachstumspakt}
\newabbreviation{EWU}{EWU}{Europäische Währungsunion}
\newabbreviation{SSM}{SSM}{Single Supervisory Mechanism}
\newabbreviation{Euro}{\euro{}}{Euro}
\newabbreviation{ESM}{ESM}{Europäischer Rettungsschirm}
\newabbreviation{Eurostat}{Eurostat}{European Statistical Office}
\newabbreviation{IWF}{IWF}{Internationaler Währungsfonds}
\newabbreviation{BIP}{BIP}{Bruttoinlandsprodukt}
\newabbreviation{IMF}{IMF}{International Monetary Fund}

% Start the document here:

\begin{document}
% Fußnoten mit Zeilenabstand 1cm
\let\footnoteOld\footnote{}
\renewcommand{\footnote}[1]{%
        \linespread{1.0}
        \footnoteOld{#1}
        \linespread{1.2}
}
%Fußnoten auf Schriftgröße 10
\setkomafont{footnote}{\footnotesize{10}}
%\footnotesize{10}

\pagenumbering{Roman} % Seitennumerierung auf römisch umstellen
\renewcommand{\refname}{Literaturverzeichnis}   % "Literatur" in
%"Literaturverzeichnis" umbenennen
\newcolumntype{C}{>{\centering\arraybackslash}X}  % Neuer Tabellen-Spalten-Typ:
%Zentriert und umbrechbar


% Titlepage
\begin{titlepage}
  \newgeometry{left=2cm, right=2cm, top=2cm, bottom=2cm}
  \begin{center}
    \textbf{Bachelor-Thesis}\\
    %\textbf{Master-Thesis}\\
    \Huge{\myTitel}\\
    \vspace{0.2cm}
  \end{center}
  \normalsize
  \vfill
  %tables must be a bite wider? or is a new line okay?
  \begin{tabbing}
    Links \= Mitte1 \= Mitte2  \= Rechts\kill
    vorgelegt von:  \> \> \> \myAutor\\
    Matr.-Nr.: \> \> \> \myMatrikelNr\\
    aus: \myLocation\\
    angefertigt im Rahmen der Bachelorprüfung \> \> \>\\
    %angefertigt im Rahmen der Masterprüfung \> \> \>\\
    für den Studiengang \> \> \>\\
    \> \> \> \myStudiengang\\
    \> \> \> am Fachbereich \myFachbereich\\
    \> \> \> der Hochschule Düsseldorf \\
    %was für ein Bindestrich wird denn fürs Datum benutzt? Ein Halbgeviertstrich oder schon ein Geviertstrich?
    Bearbeitungszeitraum: \> \> \> \\
    \> \> \> 01.01.2020 -- 30.03.2020 \\
    \> \> \> \\
    Betreuer: \> \> \> \myBetreuer\\
    Zweitprüfer: \> \> \> \myZweitpruefer\\
    %Autor: \> \> \myAutor\\
    %\> \>  Matrikelnr.: \myMatrikelNr\\
    %\> \> \myAdresse\\
    %\> \> \\
    %Abgabe: \> \> \myAbgabeDatum{}
  \end{tabbing}
\end{titlepage}

%-------Ende Titelseite-------------


% Sperrvermerk

%% !TEX root = ../../main.tex
\newpage
\thispagestyle{empty}
% lock flag
\section*{Sperrvermerk}
Die vorliegende Thesis mit dem Titel \glqq \myTitel\grqq\ enthält vertrauliche Daten des Unternehmens \myFirma.\newline
\newline
Die Thesis darf nur dem Erst- und Zweitgutachter, Mitgliedern des Prüfungsausschlusses und befugten Mitarbeitern des Studienbüros zugänglich gemacht werden. Eine Veröffentlichung und Vervielfältigung der Thesis ist -- auch in Auszügen -- nicht gestattet. Abweichende Verfahrensweisen bedürfen einer ausdrücklichen Genehmigung des Unternehmens \myFirma.\newline
\newline
Darüber hinaus werden keine Vertraulichkeitsvereinbarungen mit der Hochschule oder den Betreuern geschlossen.
\newline
\_\_\_\_\_\_\_\_\_\_\_\_\_\_\_\_\_\_\_\_\_\_\_\_ \hspace{1.5cm} \_\_\_\_\_\_\_\_\_\_\_\_\_\_\_\_\_\_\_\_\_\_\_\_ \\
(Ort, Datum)\hspace{5.9cm}
(Eigenhändige Unterschrift)

\newpage



% Inhaltsverzeichnis

\setcounter{page}{2}
\tableofcontents
\newpage


% Abkürzungsverzeichnis

%\printnomenclature
\printunsrtglossaries{} % list all entries
%\newpage

% Abbildungsverzeichnis

\listoffigures
%\newpage

% Tabellenverzeichnis

\listoftables
%\newpage

% Seitennummerierung auf arabisch und ab 1 beginnend umstellen

\pagenumbering{arabic}
\setcounter{page}{1}

% Kapitel / Inhalte

%!TEX root = master.tex
\newpage
\section{Einleitung}
Mit dieser Vorlage soll den Studierenden der \gls{hsd}\footnote{\href{https://www.hs-duesseldorf.de}{Webseite der Hochschule Düsseldorf}}eine Vorlage zur Erstellung einer Thesis mit \gls{latex} an die Hand gegeben werden, die der \gls{po} im Allgemeinen entspricht und die einfach nach den Bedürfnissen des jeweiligen betreuenden Professors angepasst werden kann.

\subsection{Grundlegender Umgang}
Diese Vorlage wurde unter einem Linux System erstellt mit Hilfe einer tex-Umgebung\footnote{\href{https://tug.org/texlive/}{Webseite der Software texlive}}. Es ist davon auszugehen, dass die Vorlage auf Windows- sowie Macsystemen ebenfalls funktioniert, hierfür erfolgt aber meinerseits keine Prüfung.

\newpage
\section{Begriffsabgrenzung}
Im folgenden möchte ich definieren, welche Maßnahmen auf welches Aufgabe der \gls{EU} es zu beurteilen gibt.

\subsection{Eurokrise}
Unter dem Begriff Eurokrise wird eine Reihe verschiedener Krisen zusammengefasst, die alle in der \gls{EU} schwelten und etwa zeitgleich das kritische Maß erreichten. Die Ursachen waren dabei jeweils unterschiedlich: in Griechenland handelte es sich vorwiegend um ein betrügerisches Staatsversagen (ZITAT), in Spanien und Irland ist eine Immobilienblase geplatzt (ZITAT), was in Irland dazu führte, das ein sowieso viel zu großes Bankwesen (ZITATE SUCHEN) kollabierte, letztes wurde auch dem Land Zypern zum Unheil, welches überwiegend durch seine volkswirtschaftliche Nähe zu Griechenland stark in Mitleidenschaft gezogen wurde (Zitat).

Im Falle Griechenlands war der EU seit 2004 bekannt, dass das Euroland hoch verschuldet war. Ein Bericht der Europäischen Kommission im Jahr 2010 kommt sogar zu dem Schluss, dass die statistischen Behörden in Griechenland die Höhe des Defizites in den Jahren 1997 bis 2003 unterschlagen hätten, vorher hatte Ende 2009 der griechische Ministerpräsident Giorgos Papandreou ein Staatsdefizit in Höhe von 12,5\% eingeräumt, nach vorherig 3,7\%
gemeldeten\footcite[vgl.][]{europaische_kommision_bericht_2010}
%Zitat http://ec.europa.eu/eurostat/documents/4187653/6406122/COM_2010_bericht_Griechenland.pdf/ Seite 30 \glqq … daraus ergab sich, dass die griechischen Statistikstellen zwischen 1997 und 2003 die Defizit- und Schuldenstandszahlen nicht korrekt gemeldet hatten. \grqq oder ich mach das als vergleich

In Spanien und Irland hingegen ist eine Immobilienblase geplatzt. Im Zuge jahrelangen starken Wachstums haben sich die Privathaushalte, häufig mit Geld aus dem Ausland, überschuldet und sind dann kollabiert. Das hat in Irland zusätzlich eine Bankenkrise hervorgerufen, da dort bis zu 80\% der neu vergebenen Kredite für Immobilien aufgenommen wurden.
Zusätzlich haben in Portugal und Italien ebenfalls marode politische Praktiken dazu geführt, dass sich eine Staatsverschuldung verschleppte. Italiens Staatsverschuldung lag 2018 noch bei 129,75\% in Relation zum Bruttoinlandsprodukts, zu Beginn der Krise 2009 immerhin schon bei 112,55\% (IWF, STATISTA).
Zypern war durch seine enge volkswirtschaftliche Verflechtung mit Griechenland in Mitleidenschaft gezogen wurden, das Rating des Landes wurde herabgestuft und ein Unglück auf einem Militärstützpunkt stürzten das Land in ein politisches Chaos (quelle?).

Im Zuge dessen haben die Märkte verstimmt reagiert und angefangen, Ländern zu misstrauen, so dass der Preis für Kredite stieg. Griechenland war somit nur das erste Land, dass um Hilfe fragen musste, um weiterhin Zugang zum Markt zu haben\footcite[vgl.][]{esm_history_nodate}% Quelle (https://www.esm.europa.eu/about-us/history#context)

\subsection{Europäischer Finanzstabilisierungsmechanismus}
Bei dem \gls{EFSM} handelt es sich um einen Vertrag, der auf Basis des Art. 122 AEU-Vertrag die \gls{EU} Kommision direkt ermächtigt, Kredite an Mitgliedsstaaten auszuzahlen, die in Schieflage geraten sind.
Die Gelder stammen dabei aus dem allgemeinen Haushaltsmitteln der Union und durchbricht somit die Regelung, dass sich die \gls{EU} nicht selbst verschulden darf. Das Risiko des Zahlungsausfalls liegt hier also bei allen Mitgliedsstaaten, also nicht nur jenen, die den \gls{Euro} als Zahlungsmittel haben.

Mit dem \gls{EFSM} stand also in erster Linie ein Instrument zur Verfügung, dass Liquidität in das Krisenland bringen konnte. Das Risiko trugen dabei alle Staaten solidarisch, auch jene, die nicht der Währungsunion beigetreten sind, da der gesamte Haushalt der Union als Sicherheitspfand eingetragen wurde. (QUELLE)

\subsection{Europäische Finanzstabilisierungsfaszilität}
Bei dem \gls{EFSF} handelt es sich, anders als beim \gls{EFSM}, um einen privatrechtlichen Vertrag zwischen den Mitgliedstaaten der Union, die der Währungsunion beigetreten sind. Diese Aktiengesellschaft wurde am 27.8.2010 nach luxemburgischem Recht, im Zuge einer Sondersitzung zur \glqq{} Eurokrise \grqq, gegründet und hatte die Aufgabe, zügig Kredite in die Krisenländer, damals Griechenland, Portugal und Irland zu bringen. Es handelte sich um eine vorübergehende Maßnahme, die nun so ziemlich
durch den \gls{ESM} abgelöst wurde. \footcite[vgl.][]{} %(http://www.efsf.europa.eu/about/operations/index.htm)
Der \gls{EFSF} hat insgesamt 174,6 Mrd. \euro{} an die drei genannten Länder geliefert und ist planmäßig nun weiterhin nur verwaltend tätig, kann aber keine neuen Kredite vergeben.

\subsection{Europäischer Stabilitätsmechanismus}
Der \gls{ESM} ist die dauerhafte Implemention der Mechanismen des \gls{EFSM} und \gls{EFSF}, wurde ähnlich wie der \gls{EFSF} als völkerrechtlicher Vertrag geschlossen und löste die vorherigen Maßnahmen im Zuge immer weiter ab.
Nur Griechenland erhielt noch einmal durch den \gls{EFSF} im Jahr 2014 ein zusätzliches Hilfspaket, durchbrach also oben genannte Zahlungsunfähigkeit, die Verwaltung des Pakets wurde jedoch durch dem \gls{ESM} übernommen.

Für den \gls{ESM} wurde vorher in der \gls{EU} die rechtliche Grundlage geschaffen, und zwar mit einem Zusatz in Art. 136 AEU-Vertrag und wurde dann in Deutschland im September 2012 rechtskräftig.

Der \gls{ESM} verfügt über ein gezeichnetes Kapital in Höhe von 704,80 Mrd. \euro{}, das eingezahlte Kapital beträgt 80,55 Mrd \euro{}. Die Finanzierung passiert über den offenen Finanzmarkt.
\begin{table}[H]
\begin{tabular}{|l|l|l|l|l|}
\hline
\rowcolor[HTML]{C0C0C0} 
{\color[HTML]{000000} Datum} & {\color[HTML]{000000} \begin{tabular}[c]{@{}l@{}}EFSF\\ in Mrd.\end{tabular}} & {\color[HTML]{000000} Fälligkeit} & {\color[HTML]{000000} \begin{tabular}[c]{@{}l@{}}EFSM\\ in Mrd.\end{tabular}} & {\color[HTML]{000000} Laufzeit (Jahre)} \\ \hline
12.01.2011 & & & 2,0 & 13 \\ \hline
12.01.2011 & & & 1,0 & 19 \\ \hline
12.01.2011 & & & 2,0 & 25 \\ \hline
01.02.2011 & 1,9 & 01.08.2032 & &  \\ \hline
01.02.2011 & 1,7 & 01.02.2033 & & \\ \hline
24.03.2011 & & & 2,4 & 14 \\ \hline
24.03.2011 & & & 1,0 & 22 \\ \hline
31.05.2011 & & & 3,0 & 10 \\ \hline
29.09.2011 & & & 2,0 & 15 \\ \hline
06.10.2011 & & & 0,5 & 22 \\ \hline
10.11.2011 & 0,9 & 01.08.2030 & & \\ \hline
10.11.2011 & 2,1 & 25.07.2031 & & \\ \hline
15.12.2011 & 1,0 & 01.08.2029 & & \\ \hline
12.01.2012 & 1,3 & 01.08.2029 & & \\ \hline
16.01.2012 & & & 1,5 & 30 \\ \hline
19.01.2012 & 0,5 & 01.07.2034 & & \\ \hline
05.03.2012 & & & 3,0 & 20 \\ \hline
03.04.2012 & 2,8 & 01.08.2031 & & \\ \hline
03.07.2012 & & & 2,3 & 15 \\ \hline
30.10.2012 & & & 1,0 & 15 \\ \hline
02.05.2013 & 0,8 & 01.08.2029 & & \\ \hline
18.06.2013  & 1,6 & 15.11.2042 & & \\ \hline
27.09.2013 & 1,0 & 27.09.2034 & & \\ \hline
04.12.2013 & 2,3 & 04.12.2033 & & \\ \hline
25.03.2014 & & & 0,8 & 10 \\ \hline
\rowcolor[HTML]{C0C0C0} 
Summe & 17,7 & & 22,5 & \\ \hline
\end{tabular}%
\caption{Die Auszahlungen des EFSF und EFSM im Überblick}
\label{Tabelle 1}
\end{table}

\begin{table}[H]
\begin{tabular}{|l|l|l|l|l|}
\hline
\rowcolor[HTML]{C0C0C0} 
\begin{tabular}[c]{@{}l@{}}Zugesagte \\ Programmvolumina (bis zu… Mrd. \euro{})\end{tabular} & \begin{tabular}[c]{@{}l@{}}Auszahlung\\ in Mrd. \euro{}\end{tabular} &       & \multicolumn{2}{l|}{\cellcolor[HTML]{C0C0C0}\begin{tabular}[c]{@{}l@{}}Rückzahlungen\\ in Mrd. \euro{}\end{tabular}} \\ \hline
\rowcolor[HTML]{EFEFEF} 
\multicolumn{3}{|l|}{\cellcolor[HTML]{EFEFEF}} & \multicolumn{2}{l|}{\cellcolor[HTML]{EFEFEF}500,0} \\ \hline
Spanien & 41,3 & -17,6 & \multicolumn{2}{l|}{23,7} \\ \hline
Zypern & 6,3 & -     & \multicolumn{2}{l|}{6,3} \\ \hline
Griechenland & 61,9 & -2,0  & \multicolumn{2}{l|}{59,9} \\ \hline
Summe & 195 & -19,6 & \multicolumn{2}{l|}{89,9} \\ \hline
\rowcolor[HTML]{C0C0C0} 
\multicolumn{3}{|l|}{\cellcolor[HTML]{C0C0C0}Verbleibendes ESM-Ausleihvolumen}                                                                                              & \multicolumn{2}{l|}{\cellcolor[HTML]{C0C0C0}410,1}                                                                   \\ \hline
\end{tabular}
\caption{Die Leistungen des ESM}
\label{Tabelle 2}
\end{table}
Insgesamt stehen dem \gls{ESM} mittlerweile folgende Werkzeuge zur Verfügung:\begin{itemize}
  \item Darlehen im Rahmen eines makroökonomischen Anpassungsprogramms\begin{itemize}
    \item In Anwendung in Irland, Portugal, Griechenland und Zypern
  \item Die Voraussetzung ist die Akzeptanz für die Umsetzung eines durch die \gls{EU}, \gls{EZB} und wo erforderlich der \gls{IWF} erstellten, mikroökonomischen Reformpaketes
  \item Ist zur Unterstützungen bei Zahlungsunfähigkeit und Marktzugangsverlust gedacht\end{itemize}
  \item Direktmarktkäufe \begin{itemize}
    \item bisher ungenutzt
    \item keine weiteren Voraussetzungen
    \item \end{itemize}
  \item Sekundärmarktkäufe\begin{itemize}
    \item bisher ungenutzt
    \item Für Mitglieder, die bereits eine andere Maßnahme des \gls{ESM} erhalten werden individuelle Konditionen erstellt\end{itemize}
  \item Vorsorgekredite\begin{itemize}
    \item bisher ungenutzt
    \item \end{itemize}
  \item Darlehen für die indirekte Rekapitalisierung von Banken\begin{itemize}
    \item In Anwendung in Spanien\end{itemize}
  \item Direkte Rekapitalisierung von Institutionen\begin{itemize}
    \item bisher ungenutzt\end{itemize}
\end{itemize}\footcite{staab_european_2013}
https://www.esm.europa.eu/assistance/lending-toolkit

\subsection{Die Hilfspakete in Zahlen}
\subsubsection{Leistungen des EFSF und EFSM}
%https://www.bundesfinanzministerium.de/Content/DE/Standardartikel/Themen/Europa/Stabilisierung_des_Euro/europaeische-finanzhilfen-efsf-efsm.html

\newpage
\section{Der Positive Einfluss des \gls{ESM} auf die Krisenländer}
Die Maßnahmen, die der \gls{ESM} dauerhaft in den Euroländern etablierte, sind nun seit bald 10 Jahren im Einsatz. Um die Entwicklung beurteilen zu können, ist ein Blick in die entsprechenden Statistiken notwendig. Im Fokus liegen dabei nur die Länder, die Kapital aus den verschiedenen Maßnahmen erhalten haben, also Griechenland, Irland, Portugal, Spanien und Zypern, auf Grund des Umfangs dieser Hausarbeit wird außerdem der Blick auf die Verschuldung, das \gls{BIP} pro Kopf und die
Arbeitslosenquote der Länder beschränkt.
\begin{figure}[H]
\begin{center}
\includegraphics[width=1\textwidth]{grid_plot}
\end{center}%https://www.imf.org/external/pubs/ft/weo/2018/01/weodata/index.aspx
%\caption[Statistik der Länder nach \gls{IMF}\footcite[Aufgerufen am 29.02.2019][man erstelle eine CSV Tabelle mit entsprechenden Quellen]{international_monetary_fund_world_2018}]
\end{figure} %\footnotetext{Aufgerufen am 29.03.2019 über https://www.imf.org/external/pubs/ft/weo/2018/01/weodata/index.aspx -- man erstelle eine CSV-Tabelle mit entsprechenden Optionen}
Abbildung 1 zeigt eine Statistik der ausgewählten Kennziffern in einem 10-Jahres-Zeitraum.
Statistik der Länder nach \gls{IMF}\footcite[Aufgerufen am 29.02.2019][man erstelle eine CSV Tabelle mit entsprechenden Quellen]{international_monetary_fund_world_2018}
\subsection{Staatsverschuldung}
\subsubsection{Griechenland}
\subsubsection{Irland}
\subsubsection{Portugal}
\subsubsection{Spanien}
\subsubsection{Zypern}
\subsection{Bruttoinlandsprodukt}
\subsubsection{Griechenland}
\subsubsection{Irland}
\subsubsection{Portugal}
\subsubsection{Spanien}
\subsubsection{Zypern}
\subsection{Arbeitslosenquote}
\subsubsection{Griechenland}
\subsubsection{Irland}
\subsubsection{Portugal}
\subsubsection{Spanien}
\subsubsection{Zypern}
%Kritiker sehen in den Reformen, die der \gls{ESM} eingeführt hat als Rechtswidrig und Falsch an. Hintergrund ist, dass der Maastricht-Vertrag von 1992 (QUELLE) vorsieht, dass die Länder der \gls{EU} untereinander nicht für die Schulden anderer Mitglieder haftet. Dennoch schaffen es die Reformpakte des \gls{ESM}, die Wirtschaft und den Haushalt wieder in Gang zu setzen.
%Betrachtet man Abbildung 1 fällt auf, dass um das Jahr 2013 jedes Land einen positiven Trend aufweist. Im Falle der Arbeitslosenstatistik zeichnet sich das Bild ganz deutlich: Griechenland schafft es, unter das Niveau im Jahre 2008 zu gelangen

%\input{kapitel/fazit/fazit}


% Literaturverzeichnis

\newpage
%\addcontentsline{toc}{section}{Literatur}

\pagenumbering{Roman} %Zähler wieder römisch ausgeben
\setcounter{page}{4}  %Zähler manuell hochsetzen

\begin{RaggedRight}
\printbibliography
\end{RaggedRight}

% Alternative Darstellung:
% Literaturverzeichnis nach Typ (@book, @arcticle ...) sortiert.
% Dazu die Zeile (\printbibliography) auskommentieren und folgenden code verwenden:

%\printbibheading
%\printbibliography[type=article,heading=subbibliography,title={Artikel}]
%\printbibliography[type=book,heading=subbibliography,title={Bücher}]
%\printbibliography[type=online,heading=subbibliography,title={Webseiten}]

%% !TEX root = ../../main.tex
\newpage
\clearpairofpagestyles
%\lehead{\includegraphics[width=3.8cm]{HSD_logo.pdf}} % this sets the picture way to high
%\lohead{\includegraphics[width=3.8cm]{HSD_logo.pdf}}
\KOMAoptions{headsepline=0pt}
\pagenumbering{gobble} % no page numbering
\includegraphics[width=5cm]{HSD_logo.pdf}
\section*{Eidesstattliche Versicherung}
\_\_\_\_\_\_\_\_\_\_\_\_\_\_\_\_\_\_\_\_\_\_\_\_ \hspace{1.5cm} \_\_\_\_\_\_\_\_\_\_\_\_\_\_\_\_\_\_\_\_\_\_\_\_ \\
\small{Name, Vorname}\hspace{5.6cm}
\small{Matrikelnummer}
\newline
Hiermit versichere ich an Eides Statt, dass ich die Bachelorarbeit/Masterarbeit (nicht Zutreffendes bitte streichen) mit dem Titel \newline
\_\_\_\_\_\_\_\_\_\_\_\_\_\_\_\_\_\_\_\_\_\_\_\_\_\_\_\_\_\_\_\_\_\_\_\_\_\_\_\_\_\_\_\_\_\_\_\_\_\_\_\_\_\_\_\_\_\_ \\
\_\_\_\_\_\_\_\_\_\_\_\_\_\_\_\_\_\_\_\_\_\_\_\_\_\_\_\_\_\_\_\_\_\_\_\_\_\_\_\_\_\_\_\_\_\_\_\_\_\_\_\_\_\_\_\_\_\_ \\
eigenständig und ohne unzulässige fremde Hilfe verfasst habe. Ich habe keine anderen als die angegebenen Quellen und Hilfsmittel benutzt und die aus fremden Quellen direkt oder indirekt übernommenen Inhalte als solche kenntlich gemacht. Für den Fall, dass die Arbeit zusätzlich auf einem Datenträger eingereicht wird, erkläre ich, dass die schriftliche und die elektronische Form vollständig übereinstimmen. Die Arbeit hat in gleicher oder ähnlicher Form noch in keinem Prüfungsverfahren vorgelegen.
Sie wurde auch nicht veröffentlicht. Ich erkläre mich damit einverstanden, dass die Arbeit mit Hilfe computergestützter Methoden auf Plagiate hin überprüft wird.\newline
\_\_\_\_\_\_\_\_\_\_\_\_\_\_\_\_\_\_\_\_\_\_\_\_ \hspace{1.5cm} \_\_\_\_\_\_\_\_\_\_\_\_\_\_\_\_\_\_\_\_\_\_\_\_ \\
\small{Ort, Datum}\hspace{5.9cm}
\small{Unterschrift}
\newline
\newline
\textbf{Belehrung:}\newline
Die vorsätzlich oder auf nur fahrlässig falsche Abgabe einer eidesstattlichen Versicherung ist strafbar:\newline
\newline
\textbf{\S\ 156 StGB - Falsche Versicherung an Eides Statt}\newline
WWer von einer zur Abnahme einer Versicherung an Eides Statt zuständigen Behörde eine solche Versicherung falsch abgibt oder unter Berufung auf eine solche Versicherung falsch aussagt, wird mit Freiheitsstrafe bis zu drei Jahren oder mit Geldstrafe bestraft.\newline
\textbf{\S\ 161 StGB - Fahrlässiger Falscheid; fahrlässige falsche Versicherung an Eides Statt}
(1) Wenn eine der in den \S\S\ 154 bis 156 bezeichneten Handlungen aus Fahrlässigkeit begangen worden ist, so tritt die Freiheitsstrafe bis zu einem Jahr oder Geldstrafe ein.
(2) Straflosigkeit tritt ein, wenn der Täter die falsche Angabe rechtzeitig berichtigt. Die Vorschriften des \S\ 158 Abs. 2 und 3 gelten entsprechend.\newline
\newline
Die vorstehende Belehrung habe ich zur Kenntnis genommen:\newline
\newline
\_\_\_\_\_\_\_\_\_\_\_\_\_\_\_\_\_\_\_\_\_\_\_\_ \hspace{1.5cm} \_\_\_\_\_\_\_\_\_\_\_\_\_\_\_\_\_\_\_\_\_\_\_\_ \\
\small{Ort, Datum}\hspace{5.9cm}
\small{Unterschrift}

\end{document}
