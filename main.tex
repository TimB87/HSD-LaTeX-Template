% !TEX root = main.tex

%%% we use arara because it makes life easier
%%% arara should work on all platforms
% arara: lualatex
% arara: xindy: {modules: [texindy, page-ranges], codepage: utf8, language: german-duden}
% arara: biber
% arara: makeglossaries
% arara: nomencl
% arara: lualatex: { shell: true, synctex: true }
% arara: lualatex: { options: [ '--synctex=1', '--shell-escape' ]}

%%% we source our input files to get a cleaner main.tex
%%% general page related settings
\documentclass[12pt,%
  %draft,%
  overfullrule=true,%
  oneside,%
  titlepage,%
  listof=totoc,%
  twoside=semi,%
  bibliography=totoc]{scrartcl} % https://www.ctan.org/pkg/scrartcl

\usepackage[automark,headsepline]{scrlayer-scrpage} % https://www.ctan.org/pkg/scrlayer-scrpage

\usepackage[a4paper,%
  left=3cm,%
  right=2.5cm,%
  top=2cm,%
  bottom=2cm]{geometry}

\usepackage{setspace} % 1,5 cm Zeilenabstand
\onehalfspacing{}

%\setlength{\parindent}{0mm} % https://komascript.de/faq_parindent
%\setlength{\parskip}{0.8em plus 0.5em minus 0.3em}
%\sloppy %Abstände variieren

%%% language settings
\usepackage{hyphsubst} % https://www.ctan.org/pkg/hyphsubst
\HyphSubstIfExists{ngerman-x-latest}{%
\HyphSubstLet{ngerman}{ngerman-x-latest}}{}
\usepackage[ngerman]{babel} % https://www.ctan.org/pkg/babel
\usepackage[babel,german=guillemets]{csquotes} % https://www.ctan.org/pkg/csquotes

%%% font settings
%\usepackage[utf8]{inputenc} % not needed with luatex
\usepackage{fontspec} % luatex needs this one
\setmainfont{Arial} % we even get to use Arial, woohoo..
\renewcommand\familydefault{\sfdefault} %defaults to helvetica
%%% if you don't or can't use luatex, use this
%% helvet is alternative to Arial for tex, or it's actually the original font
%% with minimal differences
%\usepackage[scaled=0.9]{helvet} % use together with courier package to avoid pixelated fonts
%\usepackage{courier}

% math related settings
\usepackage{amsmath} % https://www.ctan.org/pkg/amsmath
\usepackage{unicode-math} % https://www.ctan.org/pkg/unicode-math
\setmathfont{texgyrepagella-math.otf}
%\usepackage{mathptmx} % https://www.ctan.org/pkg/mathptmx -- obsolet
%\usepackage{nicefrac} % The pack­age type­sets fractions “nicely”
\usepackage{marvosym} % https://www.ctan.org/pkg/marvosym
\usepackage[hang, multiple]{footmisc} % must be placed above hyperref to avoid warnings
\setlength{\footnotemargin}{1em}
\usepackage[colorlinks=true,%
  breaklinks=true,%
  linkcolor=black,%
  citecolor=blue,%
  menucolor=black,%
  urlcolor=black]{hyperref} % adds new commands to include hyperlinks

%%% additional packages
\usepackage{ragged2e} % https://www.ctan.org/pkg/ragged2e
%\usepackage{footnote} % https://www.ctan.org/pkg/footnote
%\usepackage{fancybox} % https://www.ctan.org/pkg/fancybox

%%% table packages
\usepackage[table,xcdraw]{xcolor} % https://www.ctan.org/pkg/xcolor
\usepackage{multirow} % https://www.ctan.org/pkg/multirow
%\usepackage{array} % https://www.ctan.org/pkg/array
%\usepackage{colortbl} % https://www.ctan.org/pkg/colortbl
\usepackage{microtype} % https://www.ctan.org/pkg/microtype

%%% floats
\usepackage{float} % https://www.ctan.org/pkg/float
\usepackage{caption} % https://www.ctan.org/pkg/caption
\usepackage{graphicx} % https://www.ctan.org/pkg/graphicx

%%% list related
\usepackage{enumitem} % https://ctan.org/pkg/enumitem
%\usepackage{mdwlist} % https://www.ctan.org/pkg/mdwlist

%%% Glossary
\usepackage[%
  %sort=word,%
  xindy%
  ]{glossaries-extra} % https://www.ctan.org/pkg/glossaries
\setabbreviationstyle{long-short}
\loadglsentries{meta/gls.tex} % definitions go there
% index w/ xindy
\usepackage{makeidx} % https://www.ctan.org/pkg/makeidx
\usepackage{idxlayout} % https://www.ctan.org/pkg/idxlayout
\makeindex
%%% List of Abbreviations
\usepackage[refpage,refeq]{nomencl}
\makenomenclature

%%% Biblatex
\usepackage[%
  backend=biber,%
  style=numeric,%
  citestyle=authoryear,%
  %sortcase=true,%
  date=iso,%
  seconds=true,%
  urldate=iso,%
  %dashed=false,%
  %doi=false,%
  isbn=true,%
  %mergedate=false,%
  autocite=footnote,%
  sortlocale=de]{biblatex}
% Original Author: Andy Grunwald
% https://github.com/andygrunwald/FOM-LaTeX-Template/blob/master/skripte/weitereEbene.tex

% Optionen für Biblatex

% et al. an Stelle von u.a.
\DefineBibliographyStrings{ngerman}{%
   andothers = {{et\,al\adddot}},
}

% Klammern um das Jahr in der Fußnote
\renewbibmacro*{cite:labelyear+extrayear}{%
  \iffieldundef{labelyear}
    {}
    {\printtext[bibhyperref]{%
       \mkbibparens{%
         \printfield{labelyear}%
         \printfield{extrayear}}}}}

\DeclareNameFormat{last-first}{%
  \iffirstinits
    {\usebibmacro{name:family-given}
        {\namepartfamily}
        {\namepartgiveni}
        {\namepartprefix}
        {\namepartsuffix}
    }
    {\usebibmacro{name:family-given}
        {\namepartfamily}
        {\namepartgiven}
        {\namepartprefix}
        {\namepartsuffix}
    }%
  \usebibmacro{name:andothers}}

% Alternative Notation der Fußnoten
% Zeigt sowohl den Nachnamen als auch den Vornamen an
% Beispiel: \fullfootcite[Vgl. ][Seite 5]{Tanenbaum.2003}
\DeclareCiteCommand{\fullfootcite}[\mkbibfootnote]
  {\usebibmacro{prenote}}
  {\usebibmacro{citeindex}%
    \printnames[sortname][1-1]{author}%
    \addspace (\printfield{year})}
  {\addsemicolon\space}
  {\usebibmacro{postnote}}

%Autoren (Nachname, Vorname)
\DeclareNameAlias{default}{family-given}

%Reihenfolge von publisher, year, address verändern
% Achtung, bisher nur für den Typ @book definiert

%% Definiert @Book Eintrag
\DeclareBibliographyDriver{book}{%
  \printnames{author}%
  \newunit\addcolon\space
  \printfield{title}%
  \setunit*{,\space}%
  \printfield{edition}%
  \setunit*{\addcomma\space}%
  \printlist{publisher}%
  \newunit\newblockpunct
  \printlist{location}%
  \setunit*{\space}%
  \printfield{year}%
  \setunit*{,\space}%
  \printfield{isbn}%
  \finentry}

%% Definiert @Online Eintrag
\DeclareBibliographyDriver{online}{%
  \printnames{author}%
  \newunit\newblockpunct
  \printfield{title}%
  \setunit*{,\space}%
  %\newunit\newblock
  \printfield{url}%
  \setunit*{,\space Erscheinungsjahr:\space}%
  \printfield{year}%
  \setunit*{,\space Aufruf am:\space}%
  \printfield{note}%
  \finentry}

%% Definiert @Article Eintrag
\DeclareBibliographyDriver{article}{%
  \printnames{author}%
  \newunit\newblockpunct
  \printfield{title}%
  \setunit*{.\space In:\space}%
  %\newunit\newblock
  \usebibmacro{journal}%
  \setunit*{\space (}%
  \printfield{year}\newunit{)}%
  \finentry}

%% Definiert @InProceedings Eintrag
\DeclareBibliographyDriver{inproceedings}{%
  \printnames{author}%
  \setunit*{,\space (}%
  \printfield{year}\newunit{)}%
  \newunit\newblockpunct
  \printfield{title}%
  \setunit*{\space}%
  \usebibmacro{booktitle}%
  \setunit*{,\space}%
  \printfield{isbn}%
  \setunit*{,\space}%
  \printfield{doi}%
  \finentry}

%Doppelpunkt nach dem letzten Autor
\renewcommand*{\labelnamepunct}{\addcolon\addspace }

%Komma an Stelle des Punktes
\renewcommand*{\newunitpunct}{\addcomma\space}

%Autoren durch Semikolon trennen
\newcommand*{\bibmultinamedelim}{\addsemicolon\space}%
\newcommand*{\bibfinalnamedelim}{\addsemicolon\space}%
\AtBeginBibliography{%
  \let\multinamedelim\bibmultinamedelim
  \let\finalnamedelim\bibfinalnamedelim
}

%Titel nicht kursiv anzeigen
\DeclareFieldFormat{title}{#1\isdot}


% don't use fancyhdr to change headers/footers

% define headers and footers
%\usepackage{fancyhdr}
%\pagestyle{fancy}
%\fancyhf{}
%\fancyhead[C]{-\ \thepage\ -} % centered page numbering
%\fancyhead[L]{\leftmark}      % no footer
%\renewcommand{\headrulewidth}{0.4pt}
% footnote style
%\let\footnoteOld\footnote{}
%\renewcommand{\footnote}[1]{%
%  \linespread{1.0}%
%  \footnoteOld{#1}%
%  \linespread{1.2}%
%}
%Fußnoten auf Schriftgröße 10

% use "scrheadings"
\clearpairofpagestyles
\cfoot[\pagemark]{\pagemark}
\lehead{\headmark}
\rohead{\headmark}
\pagestyle{scrheadings}

\setkomafont{footnote}{\footnotesize{10}}

% Abkürzungsverzeichnis mit glossaries
%\makenoidxglossaries{} % w/o xindy
\makeglossaries % w/ xindy

%%% proper source code display
% disable those lines if you don't display any code
\usepackage{listings} % https://www.ctan.org/pkg/listings
% Colors for listings
\definecolor{mauve}{rgb}{0.58,0,0.82}
\definecolor{dkgreen}{rgb}{0,0.6,0}
\lstset{% general code settings
  numbers=left,%
  numberstyle=\tiny,%
  numbersep=5pt,%
  breaklines=true,%
  showstringspaces=false,%
  frame=l ,%
  xleftmargin=5pt,%
  xrightmargin=5pt,%
  basicstyle=\ttfamily\scriptsize,%
  stepnumber=1,%
  keywordstyle=\color{blue},% keyword style
  commentstyle=\color{dkgreen},% comment style
  stringstyle=\color{mauve}% string literal style
}

%input{meta/packages}
%input{meta/macros}
%input{meta/hyphenations}
%input{meta/options}

% bib resources
\addbibresource{literatur/literatur.bib}

% where to store your images
\graphicspath{{./}{./abbildungen/}}

% Weitere Ebene einfügen
%% Original Author: Andy Grunwald
% https://github.com/andygrunwald/FOM-LaTeX-Template/blob/master/skripte/weitereEbene.tex

\usepackage{titletoc}

\makeatletter

% Setze die Tiefe des Inhaltsverzeichnis auf 4 Ebenen
% Damit erscheinen \paragraph-Sektionen auch im Inhaltsverzeichnis
\setcounter{secnumdepth}{4}
\setcounter{tocdepth}{4}

% Fuege Abstand nach unten wie in einer normalen \section hinzu
% Andernfalls haette \paragraph keinen Zeilenumbruch
% Der Zeilenumbruch koennte mit einer leeren \mbox{} ersetzt werden
% Jedoch klebt dann der Text relativ nah an der Ueberschrift
\renewcommand{\paragraph}{%
  \@startsection{paragraph}{4}%
  {\z@}{3.25ex \@plus 1ex \@minus .2ex}{1.5ex plus 0.2ex}%
  {\normalfont\normalsize\bfseries\sffamily}%
}

\makeatother


% Meta information
% !TEX root = ../main.tex
%%% meta information about yourself and your working title
%%% modified version taken from:
%%% https://github.com/andygrunwald/FOM-LaTeX-Template/blob/master/skripte/meta.tex

% author
\newcommand{\myAutor}{Tim Biermann}

% address
\newcommand{\myAdresse}{Münsterstra\ss{}e 156 \\ \> \> 40476 Düsseldorf}
\newcommand{\myLocation}{Düsseldorf}

% titel
\newcommand{\myTitel}{\LaTeX\ Vorlage für die Thesis}
\newcommand{\mySubtitel}{or: How I Learned to Stop Worrying and Love the Bomb}

% supervising tutors
\newcommand{\myBetreuer}{Prof.\ Dr.\ Erika Mustermann}
\newcommand{\myZweitpruefer}{Prof.\ Dr.\ Peter Parker}

% faculty
\newcommand{\myFachbereich}{Wirtschaftswissenschaften}

% type
\newcommand{\myLehrveranstaltung}{Thesis}

% matrikelnumber
\newcommand{\myMatrikelNr}{123456}

% city
\newcommand{\myOrt}{Düsseldorf}

% deadline
\newcommand{\myAbgabeDatum}{21.03.2019}

% your semester
\newcommand{\mySemesterZahl}{7}

% name university
\newcommand{\myHochschulName}{Hochschule Düsseldorf}

% course of studies
\newcommand{\myStudiengang}{Bachelor Business Administration}

% company
\newcommand{\myFirma}{Meine Wunschfirma GmbH}


% PDF Meta Data
\hypersetup{%
  pdfinfo={%
    Title={\myTitel},%
    Subject={\myStudiengang},%
    Author={\myAutor},%
    Build=1.1%
  }%
}

\begin{document}

\pagenumbering{Roman} % Roman page numbering style
\renewcommand{\refname}{Literaturverzeichnis} % rename "Literatur" to "Literaturverzeichnis"

% !TEX root = ../../main.tex
% a new table style, centered and wrapable
%\newcolumntype{C}{>{\centering\arraybackslash}X}
\begin{titlepage}
  \newgeometry{left=2cm, right=2cm, top=2cm, bottom=2cm}
  \begin{center}
  \textbf{Bachelor-Thesis:}\\
  %\textbf{Master-Thesis}\\ % or if you are on your master thesis..
  \Huge{\myTitel}\\
  \vspace{0.2cm}
  \end{center}
  \normalsize
  \vfill
  \begin{tabbing}
    10\% \= 10\% \= 10\% \= 10\% \= 10\% \= 10\% \= 10\% \= 10\% \= 10\% \= 10\%\kill
    vorgelegt von: \> \> \> \> \> \> \> \> \> \myAutor\\
    Matr.-Nr.: \> \> \> \> \> \> \> \> \> \myMatrikelNr\\
    aus: \> \> \> \> \> \> \> \> \> \myLocation\\
    angefertigt im Rahmen der Bachelorprüfung\\%   \\ % same as above
    %angefertigt im Rahmen der Masterprüfung\\
    für den Studiengang \myStudiengang \\% am Fachbereich \myFachbereich der Hochschule Düsseldorf  \\
    am Fachbereich \myFachbereich\ der \myHochschulName \\
    Bearbeitungszeitraum: \> \> \> \> \> \> \> \> \> 01.01.2020 -- 30.03.2020 \\
    \\
    Betreuer/in: \> \> \> \> \> \> \> \> \> \myBetreuer\\
    Zweite/r Prüfer/in: \> \> \> \> \> \> \> \> \> \myZweitpruefer\\
  \end{tabbing}
\end{titlepage}
 % Titlepage

%% !TEX root = ../../main.tex
\newpage
\thispagestyle{empty}
% lock flag
\section*{Sperrvermerk}
Die vorliegende Thesis mit dem Titel \glqq \myTitel\grqq\ enthält vertrauliche Daten des Unternehmens \myFirma.\newline
\newline
Die Thesis darf nur dem Erst- und Zweitgutachter, Mitgliedern des Prüfungsausschlusses und befugten Mitarbeitern des Studienbüros zugänglich gemacht werden. Eine Veröffentlichung und Vervielfältigung der Thesis ist -- auch in Auszügen -- nicht gestattet. Abweichende Verfahrensweisen bedürfen einer ausdrücklichen Genehmigung des Unternehmens \myFirma.\newline
\newline
Darüber hinaus werden keine Vertraulichkeitsvereinbarungen mit der Hochschule oder den Betreuern geschlossen.
\newline
\_\_\_\_\_\_\_\_\_\_\_\_\_\_\_\_\_\_\_\_\_\_\_\_ \hspace{1.5cm} \_\_\_\_\_\_\_\_\_\_\_\_\_\_\_\_\_\_\_\_\_\_\_\_ \\
(Ort, Datum)\hspace{4.9cm}
(Eigenhändige Unterschrift)

\newpage
 % lock flag

%%% table of content
\setcounter{page}{2}
\begin{sloppypar} % prevents overfull \hbox in ToC
\tableofcontents
\newpage

%%% list of abbreviations
\renewcommand{\nomname}{Abkürzungsverzeichnis}
\printnomenclature*
\newpage

% table of figures
\listoffigures
\newpage

% table of tables
\listoftables
\end{sloppypar} % doesn't touch the rest of our precious document
\newpage

\pagenumbering{arabic} % Arabic page numbering style
\setcounter{page}{1} % reset page counter

% this is where you write your text
%!TEX root = master.tex
\newpage
\section{Einleitung}
Mit dieser Vorlage soll den Studierenden der \gls{hsd}\footnote{\href{https://www.hs-duesseldorf.de}{Webseite der Hochschule Düsseldorf}}eine Vorlage zur Erstellung einer Thesis mit \gls{latex} an die Hand gegeben werden, die der \gls{po} im Allgemeinen entspricht und die einfach nach den Bedürfnissen des jeweiligen betreuenden Professors angepasst werden kann.

\subsection{Grundlegender Umgang}
Diese Vorlage wurde unter einem Linux System erstellt mit Hilfe einer tex-Umgebung\footnote{\href{https://tug.org/texlive/}{Webseite der Software texlive}}. Es ist davon auszugehen, dass die Vorlage auf Windows- sowie Macsystemen ebenfalls funktioniert, hierfür erfolgt aber meinerseits keine Prüfung.

\newpage
\section{Begriffsabgrenzung}
Im folgenden möchte ich definieren, welche Maßnahmen auf welches Aufgabe der \gls{EU} es zu beurteilen gibt.

\subsection{Eurokrise}
Unter dem Begriff Eurokrise wird eine Reihe verschiedener Krisen zusammengefasst, die alle in der \gls{EU} schwelten und etwa zeitgleich das kritische Maß erreichten. Die Ursachen waren dabei jeweils unterschiedlich: in Griechenland handelte es sich vorwiegend um ein betrügerisches Staatsversagen (ZITAT), in Spanien und Irland ist eine Immobilienblase geplatzt (ZITAT), was in Irland dazu führte, das ein sowieso viel zu großes Bankwesen (ZITATE SUCHEN) kollabierte, letztes wurde auch dem Land Zypern zum Unheil, welches überwiegend durch seine volkswirtschaftliche Nähe zu Griechenland stark in Mitleidenschaft gezogen wurde (Zitat).

Im Falle Griechenlands war der EU seit 2004 bekannt, dass das Euroland hoch verschuldet war. Ein Bericht der Europäischen Kommission im Jahr 2010 kommt sogar zu dem Schluss, dass die statistischen Behörden in Griechenland die Höhe des Defizites in den Jahren 1997 bis 2003 unterschlagen hätten, vorher hatte Ende 2009 der griechische Ministerpräsident Giorgos Papandreou ein Staatsdefizit in Höhe von 12,5\% eingeräumt, nach vorherig 3,7\%
gemeldeten\footcite[vgl.][]{europaische_kommision_bericht_2010}
%Zitat http://ec.europa.eu/eurostat/documents/4187653/6406122/COM_2010_bericht_Griechenland.pdf/ Seite 30 \glqq … daraus ergab sich, dass die griechischen Statistikstellen zwischen 1997 und 2003 die Defizit- und Schuldenstandszahlen nicht korrekt gemeldet hatten. \grqq oder ich mach das als vergleich

In Spanien und Irland hingegen ist eine Immobilienblase geplatzt. Im Zuge jahrelangen starken Wachstums haben sich die Privathaushalte, häufig mit Geld aus dem Ausland, überschuldet und sind dann kollabiert. Das hat in Irland zusätzlich eine Bankenkrise hervorgerufen, da dort bis zu 80\% der neu vergebenen Kredite für Immobilien aufgenommen wurden.
Zusätzlich haben in Portugal und Italien ebenfalls marode politische Praktiken dazu geführt, dass sich eine Staatsverschuldung verschleppte. Italiens Staatsverschuldung lag 2018 noch bei 129,75\% in Relation zum Bruttoinlandsprodukts, zu Beginn der Krise 2009 immerhin schon bei 112,55\% (IWF, STATISTA).
Zypern war durch seine enge volkswirtschaftliche Verflechtung mit Griechenland in Mitleidenschaft gezogen wurden, das Rating des Landes wurde herabgestuft und ein Unglück auf einem Militärstützpunkt stürzten das Land in ein politisches Chaos (quelle?).

Im Zuge dessen haben die Märkte verstimmt reagiert und angefangen, Ländern zu misstrauen, so dass der Preis für Kredite stieg. Griechenland war somit nur das erste Land, dass um Hilfe fragen musste, um weiterhin Zugang zum Markt zu haben\footcite[vgl.][]{esm_history_nodate}% Quelle (https://www.esm.europa.eu/about-us/history#context)

\subsection{Europäischer Finanzstabilisierungsmechanismus}
Bei dem \gls{EFSM} handelt es sich um einen Vertrag, der auf Basis des Art. 122 AEU-Vertrag die \gls{EU} Kommision direkt ermächtigt, Kredite an Mitgliedsstaaten auszuzahlen, die in Schieflage geraten sind.
Die Gelder stammen dabei aus dem allgemeinen Haushaltsmitteln der Union und durchbricht somit die Regelung, dass sich die \gls{EU} nicht selbst verschulden darf. Das Risiko des Zahlungsausfalls liegt hier also bei allen Mitgliedsstaaten, also nicht nur jenen, die den \gls{Euro} als Zahlungsmittel haben.

Mit dem \gls{EFSM} stand also in erster Linie ein Instrument zur Verfügung, dass Liquidität in das Krisenland bringen konnte. Das Risiko trugen dabei alle Staaten solidarisch, auch jene, die nicht der Währungsunion beigetreten sind, da der gesamte Haushalt der Union als Sicherheitspfand eingetragen wurde. (QUELLE)

\subsection{Europäische Finanzstabilisierungsfaszilität}
Bei dem \gls{EFSF} handelt es sich, anders als beim \gls{EFSM}, um einen privatrechtlichen Vertrag zwischen den Mitgliedstaaten der Union, die der Währungsunion beigetreten sind. Diese Aktiengesellschaft wurde am 27.8.2010 nach luxemburgischem Recht, im Zuge einer Sondersitzung zur \glqq{} Eurokrise \grqq, gegründet und hatte die Aufgabe, zügig Kredite in die Krisenländer, damals Griechenland, Portugal und Irland zu bringen. Es handelte sich um eine vorübergehende Maßnahme, die nun so ziemlich
durch den \gls{ESM} abgelöst wurde. \footcite[vgl.][]{} %(http://www.efsf.europa.eu/about/operations/index.htm)
Der \gls{EFSF} hat insgesamt 174,6 Mrd. \euro{} an die drei genannten Länder geliefert und ist planmäßig nun weiterhin nur verwaltend tätig, kann aber keine neuen Kredite vergeben.

\subsection{Europäischer Stabilitätsmechanismus}
Der \gls{ESM} ist die dauerhafte Implemention der Mechanismen des \gls{EFSM} und \gls{EFSF}, wurde ähnlich wie der \gls{EFSF} als völkerrechtlicher Vertrag geschlossen und löste die vorherigen Maßnahmen im Zuge immer weiter ab.
Nur Griechenland erhielt noch einmal durch den \gls{EFSF} im Jahr 2014 ein zusätzliches Hilfspaket, durchbrach also oben genannte Zahlungsunfähigkeit, die Verwaltung des Pakets wurde jedoch durch dem \gls{ESM} übernommen.

Für den \gls{ESM} wurde vorher in der \gls{EU} die rechtliche Grundlage geschaffen, und zwar mit einem Zusatz in Art. 136 AEU-Vertrag und wurde dann in Deutschland im September 2012 rechtskräftig.

Der \gls{ESM} verfügt über ein gezeichnetes Kapital in Höhe von 704,80 Mrd. \euro{}, das eingezahlte Kapital beträgt 80,55 Mrd \euro{}. Die Finanzierung passiert über den offenen Finanzmarkt.
\begin{table}[H]
\begin{tabular}{|l|l|l|l|l|}
\hline
\rowcolor[HTML]{C0C0C0} 
{\color[HTML]{000000} Datum} & {\color[HTML]{000000} \begin{tabular}[c]{@{}l@{}}EFSF\\ in Mrd.\end{tabular}} & {\color[HTML]{000000} Fälligkeit} & {\color[HTML]{000000} \begin{tabular}[c]{@{}l@{}}EFSM\\ in Mrd.\end{tabular}} & {\color[HTML]{000000} Laufzeit (Jahre)} \\ \hline
12.01.2011 & & & 2,0 & 13 \\ \hline
12.01.2011 & & & 1,0 & 19 \\ \hline
12.01.2011 & & & 2,0 & 25 \\ \hline
01.02.2011 & 1,9 & 01.08.2032 & &  \\ \hline
01.02.2011 & 1,7 & 01.02.2033 & & \\ \hline
24.03.2011 & & & 2,4 & 14 \\ \hline
24.03.2011 & & & 1,0 & 22 \\ \hline
31.05.2011 & & & 3,0 & 10 \\ \hline
29.09.2011 & & & 2,0 & 15 \\ \hline
06.10.2011 & & & 0,5 & 22 \\ \hline
10.11.2011 & 0,9 & 01.08.2030 & & \\ \hline
10.11.2011 & 2,1 & 25.07.2031 & & \\ \hline
15.12.2011 & 1,0 & 01.08.2029 & & \\ \hline
12.01.2012 & 1,3 & 01.08.2029 & & \\ \hline
16.01.2012 & & & 1,5 & 30 \\ \hline
19.01.2012 & 0,5 & 01.07.2034 & & \\ \hline
05.03.2012 & & & 3,0 & 20 \\ \hline
03.04.2012 & 2,8 & 01.08.2031 & & \\ \hline
03.07.2012 & & & 2,3 & 15 \\ \hline
30.10.2012 & & & 1,0 & 15 \\ \hline
02.05.2013 & 0,8 & 01.08.2029 & & \\ \hline
18.06.2013  & 1,6 & 15.11.2042 & & \\ \hline
27.09.2013 & 1,0 & 27.09.2034 & & \\ \hline
04.12.2013 & 2,3 & 04.12.2033 & & \\ \hline
25.03.2014 & & & 0,8 & 10 \\ \hline
\rowcolor[HTML]{C0C0C0} 
Summe & 17,7 & & 22,5 & \\ \hline
\end{tabular}%
\caption{Die Auszahlungen des EFSF und EFSM im Überblick}
\label{Tabelle 1}
\end{table}

\begin{table}[H]
\begin{tabular}{|l|l|l|l|l|}
\hline
\rowcolor[HTML]{C0C0C0} 
\begin{tabular}[c]{@{}l@{}}Zugesagte \\ Programmvolumina (bis zu… Mrd. \euro{})\end{tabular} & \begin{tabular}[c]{@{}l@{}}Auszahlung\\ in Mrd. \euro{}\end{tabular} &       & \multicolumn{2}{l|}{\cellcolor[HTML]{C0C0C0}\begin{tabular}[c]{@{}l@{}}Rückzahlungen\\ in Mrd. \euro{}\end{tabular}} \\ \hline
\rowcolor[HTML]{EFEFEF} 
\multicolumn{3}{|l|}{\cellcolor[HTML]{EFEFEF}} & \multicolumn{2}{l|}{\cellcolor[HTML]{EFEFEF}500,0} \\ \hline
Spanien & 41,3 & -17,6 & \multicolumn{2}{l|}{23,7} \\ \hline
Zypern & 6,3 & -     & \multicolumn{2}{l|}{6,3} \\ \hline
Griechenland & 61,9 & -2,0  & \multicolumn{2}{l|}{59,9} \\ \hline
Summe & 195 & -19,6 & \multicolumn{2}{l|}{89,9} \\ \hline
\rowcolor[HTML]{C0C0C0} 
\multicolumn{3}{|l|}{\cellcolor[HTML]{C0C0C0}Verbleibendes ESM-Ausleihvolumen}                                                                                              & \multicolumn{2}{l|}{\cellcolor[HTML]{C0C0C0}410,1}                                                                   \\ \hline
\end{tabular}
\caption{Die Leistungen des ESM}
\label{Tabelle 2}
\end{table}
Insgesamt stehen dem \gls{ESM} mittlerweile folgende Werkzeuge zur Verfügung:\begin{itemize}
  \item Darlehen im Rahmen eines makroökonomischen Anpassungsprogramms\begin{itemize}
    \item In Anwendung in Irland, Portugal, Griechenland und Zypern
  \item Die Voraussetzung ist die Akzeptanz für die Umsetzung eines durch die \gls{EU}, \gls{EZB} und wo erforderlich der \gls{IWF} erstellten, mikroökonomischen Reformpaketes
  \item Ist zur Unterstützungen bei Zahlungsunfähigkeit und Marktzugangsverlust gedacht\end{itemize}
  \item Direktmarktkäufe \begin{itemize}
    \item bisher ungenutzt
    \item keine weiteren Voraussetzungen
    \item \end{itemize}
  \item Sekundärmarktkäufe\begin{itemize}
    \item bisher ungenutzt
    \item Für Mitglieder, die bereits eine andere Maßnahme des \gls{ESM} erhalten werden individuelle Konditionen erstellt\end{itemize}
  \item Vorsorgekredite\begin{itemize}
    \item bisher ungenutzt
    \item \end{itemize}
  \item Darlehen für die indirekte Rekapitalisierung von Banken\begin{itemize}
    \item In Anwendung in Spanien\end{itemize}
  \item Direkte Rekapitalisierung von Institutionen\begin{itemize}
    \item bisher ungenutzt\end{itemize}
\end{itemize}\footcite{staab_european_2013}
https://www.esm.europa.eu/assistance/lending-toolkit

\subsection{Die Hilfspakete in Zahlen}
\subsubsection{Leistungen des EFSF und EFSM}
%https://www.bundesfinanzministerium.de/Content/DE/Standardartikel/Themen/Europa/Stabilisierung_des_Euro/europaeische-finanzhilfen-efsf-efsm.html

\newpage
\section{Spaß mit LaTeX}
Expand $(a+b)^n$:
\newcount\mycntr

  \begin{center}
    \mycntr=0
    \loop\advance\mycntr by 1
    \ifnum\mycntr<20
      $(a\hskip\mycntr pt +\hskip\mycntr pt b)^n$\\
    \repeat
  \end{center}

If~~$\displaystyle\lim_{x\rightarrow8}\frac{1}{x{-}8}=\infty$
~~then~~$\displaystyle\lim_{x\rightarrow5}\frac{1}{x{-}5}=\rotatebox{90}{5}$

%% !TEX root = ../main.tex
\section{Fazit}
Wünsche Euch allen viel Erfolg für das 7. Semester und bei der Erstellung der Thesis. Über Anregungen und Verbesserung an dieser Vorlage würde ich mich sehr freuen. 


\newpage % bibliography
\addcontentsline{toc}{section}{Literatur}
\pagenumbering{Roman} % Change back to Roman page numbering style
\setcounter{page}{6}  % Change the number accordingly

%%% print index
\addcontentsline{toc}{section}{Index}
\printindex
\newpage

%%% Glossaries
%\printunsrtglossaries{} % list all entries w/o xindy
\printglossaries{} % list all entries w/ xindy
\newpage

%\begin{RaggedRight} % print text ragged right
\printbibliography
\newpage
%\end{RaggedRight}
%%% Alternative Darstellung:
%%% Literaturverzeichnis nach Typ (@book, @arcticle ...) sortiert.
%%% Dazu die Zeile (\printbibliography) auskommentieren und folgenden code verwenden:

%\printbibheading
%\printbibliography[type=article,heading=subbibliography,title={Artikel}]
%\printbibliography[type=book,heading=subbibliography,title={Bücher}]
%\printbibliography[type=online,heading=subbibliography,title={Webseiten}]

%\newpage
\pagenumbering{gobble} % Keine Seitenzahlen mehr

%-----------------------------------
% Ehrenwörtliche Erklärung
%-----------------------------------
\section*{Ehrenwörtliche Erklärung}
Hiermit versichere ich, dass die vorliegende Arbeit von mir selbstständig und ohne unerlaubte Hilfe angefertigt worden ist, insbesondere dass ich alle Stellen, die wörtlich oder annähernd wörtlich aus Veröffentlichungen entnommen sind, durch Zitate als solche gekennzeichnet habe. Ich versichere auch, dass die von mir eingereichte schriftliche Version mit der digitalen Version übereinstimmt. Weiterhin erkläre ich, dass die Arbeit in gleicher oder ähnlicher Form noch keiner Prüfungsbehörde/Prüfungsstelle vorgelegen hat. Ich erkläre mich damit \textcolor{red}{einverstanden/nicht} einverstanden, dass die Arbeit der Öffentlichkeit zugänglich gemacht wird. Ich erkläre mich damit einverstanden, dass die Digitalversion dieser Arbeit zwecks Plagiatsprüfung auf die Server externer Anbieter hoch geladen werden darf. Die Plagiatsprüfung stellt keine Zurverfügungstellung für die Öffentlichkeit dar.

\par\medskip
\par\medskip

\vspace{5cm}

\begin{table}[H]
	\centering
	\begin{tabular*}{\textwidth}{c @{\extracolsep{\fill}} ccccc}
		\myOrt, \today
		&
		% Hinterlege deine eingescannte Unterschrift im Verzeichnis /abbildungen und nenne sie unterschrift.png
		% Bilder mit transparentem Hintergrund können teils zu Problemen führen
		\includegraphics[width=0.35\textwidth]{unterschrift}\vspace*{-0.35cm}
		\\
		\rule[0.5ex]{12em}{0.55pt} & \rule[0.5ex]{12em}{0.55pt} \\
		(Ort, Datum) & (Eigenhändige Unterschrift)
		\\
	\end{tabular*} \\
\end{table}

\end{document}
