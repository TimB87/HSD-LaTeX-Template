% !TEX root = ../../main.tex
% Kapitel

\section{Tipps zu häufig gemachten Fehlern}

\subsection{Abbildungen, Tabellen, Listings, etc.}
\begin{enumerate}
 \item Die Schriftgröße von Text in Abbildungen muss sich nach der Schriftgröße
 des regulären Textes richten.
 \item Alle Abbildungen, Tabellen, Listings, etc. sind mit einer Beschriftung
 und Nummerierung zu versehen. Im Text muss mit Hilfe der Nummerierung auf
 die jeweilige Abbildung, Tabelle bzw. das Listing, etc. verwiesen und eine
 Erläuterung der Abbildung, Tabelle bzw. des Listings verfasst werden.
\end{enumerate}

\subsection{Text}
\begin{enumerate}
 \item Abkürzungen werden einmalig wie in Abschnitt~\ref{sec:acros} beschrieben eingeführt
 und verwendet.
 \item Fachbegriffe müssen eingeführt und definiert werden. Der Fachbegriff kann
 z. B. einmal \textit{kursiv} gedruckt und danach normal geschrieben werden. Für
 die Definition und Erklärung sollte einschlägige Literatur verwendet werden.
 \item Es muss eine Rechtschreib- und Grammatikprüfung verwendet werden.
 \item Es sollte eine Korrektur durch Dritte durchgeführt werden.
 \item Es muss Groß-/Kleinschreibung im Literaturverzeichnis beachtet werden.
 \item Es müssen Deutsche Anführungsstriche verwendet werden: \glqq \dots\grqq
\end{enumerate}

\subsection{Diverses}
\begin{enumerate}
 \item Wenn es sich bei der Arbeit um einen Angriff dreht, dann muss
 (am Besten am Beginn der Arbeit) die Hackerethik zusammenfassend beschrieben
 und dabei konkret auf den Angriff bezogen werden.
 \item Internetquellen sollen nicht in das Literaturverzeichnis, sondern über
 eine Fußnote unter Angabe der URL und dem letzten Abrufdatum dokumentiert
 werden.
\end{enumerate}
