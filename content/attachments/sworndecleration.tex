%%% HSD-LaTeX-Template (c) by Tim Biermann
%%%
%%% HSD-LaTeX-Template is licensed under a
%%% Creative Commons Attribution-ShareAlike 4.0 International License.
%%%
%%% You should have received a copy of the license along with this
%%% work. If not, see <http://creativecommons.org/licenses/by-sa/4.0/>.

% !TEX root = ../../main.tex
\newpage
\clearpairofpagestyles
\KOMAoptions{headsepline=0pt}
\pagenumbering{gobble} % no page numbering
\includegraphics[width=5cm]{HSD_logo.pdf}
\section*{Eidesstattliche Versicherung}
\rule{6cm}{1pt} \hspace{1.23cm} \rule{6cm}{1pt}\\
\footnotesize{Name, Vorname}\hspace{4.8cm}
\footnotesize{Matrikelnummer}
\newline\newline
\newline
Hiermit versichere ich an Eides Statt, dass ich die Bachelorarbeit/Masterarbeit (nicht Zutreffendes bitte streichen) mit dem Titel \newline
\rule{\textwidth}{1pt}\\
\rule{\textwidth}{1pt}\\
eigenständig und ohne unzulässige fremde Hilfe verfasst habe. Ich habe keine anderen als die angegebenen Quellen und Hilfsmittel benutzt und die aus fremden Quellen direkt oder indirekt übernommenen Inhalte als solche kenntlich gemacht. Für den Fall, dass die Arbeit zusätzlich auf einem Datenträger eingereicht wird, erkläre ich, dass die schriftliche und die elektronische Form vollständig übereinstimmen. Die Arbeit hat in gleicher oder ähnlicher Form noch in keinem Prüfungsverfahren vorgelegen.
Sie wurde auch nicht veröffentlicht. Ich erkläre mich damit einverstanden, dass die Arbeit mit Hilfe computergestützter Methoden auf Plagiate hin überprüft wird.\newline
\rule{6cm}{1pt} \hspace{1.21cm} \rule{6cm}{1pt}\\
\newline
\footnotesize{Ort, Datum}\hspace{5.55cm}
\footnotesize{Unterschrift}
\newline\newline
\textbf{Belehrung:}\newline
Die vorsätzlich oder auf nur fahrlässig falsche Abgabe einer eidesstattlichen Versicherung ist strafbar:\newline
\newline
\textbf{\S\ 156 StGB - Falsche Versicherung an Eides Statt}\newline
Wer von einer zur Abnahme einer Versicherung an Eides Statt zuständigen Behörde eine solche Versicherung falsch abgibt oder unter Berufung auf eine solche Versicherung falsch aussagt, wird mit Freiheitsstrafe bis zu drei Jahren oder mit Geldstrafe bestraft.\newline\newline
\textbf{\S\ 161 StGB - Fahrlässiger Falscheid; fahrlässige falsche Versicherung an Eides Statt}\\
(1) Wenn eine der in den \S\S\ 154 bis 156 bezeichneten Handlungen aus Fahrlässigkeit begangen worden ist, so tritt die Freiheitsstrafe bis zu einem Jahr oder Geldstrafe ein.
(2) Straflosigkeit tritt ein, wenn der Täter die falsche Angabe rechtzeitig berichtigt. Die Vorschriften des \S\ 158 Abs. 2 und 3 gelten entsprechend.\newline
\newline
Die vorstehende Belehrung habe ich zur Kenntnis genommen:\newline
\newline
\rule{6cm}{1pt} \hspace{1.21cm} \rule{6cm}{1pt}\\
\footnotesize{Ort, Datum}\hspace{5.55cm}
\footnotesize{Unterschrift}
